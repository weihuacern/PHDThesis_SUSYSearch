%% uctest.tex 11/3/94
%% Copyright (C) 1988-2004 Daniel Gildea, BBF, Ethan Munson.
%
% This work may be distributed and/or modified under the
% conditions of the LaTeX Project Public License, either version 1.3
% of this license or (at your option) any later version.
% The latest version of this license is in
%   http://www.latex-project.org/lppl.txt
% and version 1.3 or later is part of all distributions of LaTeX
% version 2003/12/01 or later.
%
% This work has the LPPL maintenance status "maintained".
% 
% The Current Maintainer of this work is Daniel Gildea.
%
% 2007/08/01
% LaTeX Package "ucr" is modified from LaTeX package "ucthesis."
% This modification is therefore under to the conditions of 
% the LaTeX Project Public License.
% Its formality is suitable for the dissertation of University of
% California, Riverside.
% This test document is for the convenience of all students of
% University of California, Riverside.
% Contact Charles Yang at chcyang@yahoo.com if you like.
% Charles Yang has nothing to do with the original author's sarcasm.
%
%\documentclass[11pt]{ucthesis}
%\documentclass[11pt]{ucr}
\documentclass[oneside,final,letterpaper]{ucr}
\def\dsp{\def\baselinestretch{2.0}\large\normalsize}
\dsp
%\documentclass[11pt]{book}
\begin{document}
%Declarations for Front Matter

\title{Search for Supersymmetry in Proton-Proton Collisions at a Center-of-Mass Energy of 13 TeV with a Customized Top Quark Tagger in the All-Hadronic Final State}
\author{Hua Wei}
\degreemonth{December}
\degreeyear{2017}
\degree{Doctor of Philosophy}
\chair{Prof. John William Gary}
\othermembers{Prof. Haibo Yu\\Prof. Owen Long}
\numberofmembers{3}
\field{Physics}
%\field{Experimental High Energy Particle Physics}
\campus{Riverside}

\maketitle
\copyrightpage{}
\approvalpage{}

\degreesemester{Fall}
\begin{frontmatter}

\begin{acknowledgements}
Now it is the end of this adventure. It is a just a small one, comparing with the big world. But, I can hardly survive without the supports of my colleagues and friends. I owe a lot of thanks since I left CERN in a hurry. 

First, I would like to give a big thank you to my advisor, Bill. I can hardly to have such a good chance to work at CERN, with CMS HCAL guys and on SUSY analysis, without his support. He also put huge efforts to help me to improve the quality of thesis, which I am really appreciated. He helps a lot on both academic work and real life issue. Our UCR sub-group meeting, hosting by Bill and Owen on weekly basis, is a great place to share working progress and ideas. I benefit a lot from the reports and discussions from Florent, Amin and Georgia in the meeting. 

Working on CMS detector is not easy. A new member working on the detector is just like a new swimmer struggling in the sea. But, I am lucky enough to have some nice guys supporting me when I am drowning. Jared is the guy who helps me to start the HCAL related work at very beginning. I work under Dick's guidance after Jared left. Dick is a real expert on HCAL. It is indeed a great honor to work with him. I hope I can be a person like him in future: leading the work, helping young people. Josh, I cannot remember how many times we work together to deal with HCAL in the midnight. I bet you will miss me also, especially when you are trying to find someone to practice Chinese in the intervals of operation work. Edmund helps a lot on dealing with the damn HO mapping. I was impressed by his extremely high profile skill in coding. Ted L helps me a lot on the map fiber id checking. His cool head and professional attitude is always valuable. Seth, John and Martin, I will never forget the days we work together: we are waked up in the mid-night, debugging on the HCAL, and finally fixed the issue. 

The analysis work is also challenging, especially to a beginner. However, I am just so lucky to have Florent and Hongxuan as analysis conveners. They answered tons of rudimental questions from me when I was starting the analysis work. They also gave me a lot of good advises when I was getting stuck on work. Kenichi's serious and tireless attitudes on the work are highly respectful in the group. I benefit a lot from his wide knowledge on both physics analysis and detector. It is a great honor to work with all the stop team alpha members: Zhenbin, Joe, Nadja, Scarlet, Koushik, Joshi, Hui, etc. I indeed learned a lot from your folks. 

Life in Geneva is wonderful, peaceful and productive. However, it was a bit sad to leave my friend in Riverside and move to Geneva at the beginning. Mingjian and Peng are my primary roommates at CERN. I knew them since I was an undergraduate student. It was a great reunion, until they passed away in a car accident in 2015. There are lots of words I want to but have no chance to tell them. 

Fortunately I still have friends to support each other, in both daily life and work. Menglei is such a great friend. It is always pleasant when talking with her, on both daily life and physics topics (ECAL, DQM, SUSY, etc.). Manuel is my last office mate. I was lucky to catch his defense before he move on to his new post-doc job. Huilin is a friend and an ``enemy". He works on stop search team A, a great competitor. We are did not exchange much idea on physics until my last year. I benefit a lot from him on the discussion of convolutional neural network. Hualin is a reliable friend. It is always a pleasure to chat with him on various topics, like CSC, Higgs to ZZ 4 lepton analysis, etc. He also help on to babysitter my car when I left the town. 

One of the most interesting things at CERN is that you can always find someone to have physics night after party. I always remember Qi (Atlas), Xiangyang (Atlas), and Jingyu (CMS) had such a night together. We discuss various high-energy physics topics (and drinking) until 4:00 a.m. in the morning. I would also give a special thanks to Jingyu, for hosting me when I visit fermilab, it was indeed a good night for me (only) to watch Cavalry beat Warrior in the buffalo chicken wing, haha. 

Yali is my dearest friend. She always covered my back when I was on fire. Her professional knowledge on feminism (although I never take the whole package, haha) and other aspects of expertise in social science give me a brand new view of the world. I never feel tired when talking with her. 

Finally, my deepest appreciation belongs to my family, especially to my parents. Sometimes, it is just so supportive to know that there is a place I can always come back. 
\end{acknowledgements}
\begin{dedication}
\null\vfil
{\large
\begin{center}
%\usepackage{CJK}
%\begin{CJK}{GBK}{song}
To my beloved grandpa, who always encourages me to pursue science. But I lost him in my second year of Ph.D.
%中文测试
%\end{CJK}
\end{center}
}
\vfil\null
\end{dedication}
\begin{abstract}
	This thesis presents a search for direct and gluino-mediated production of supersymmetric scalar top-quark pairs in the all-hadronic final state using top quark tagging. The results of search are based on a sample of proton-proton collision events collected at a center-of-mass energy of 13 TeV with the CMS detector at the CERN Large Hadron Collider, corresponding to an integrated luminosity of 36 fb$^{-1}$. The results of the search are interpreted in several simplified models for supersymmetric particle production.
\end{abstract}
\tableofcontents
\listoffigures
\listoftables
\end{frontmatter}

%\part{First Part}
%introduce the common definition files
%c4 Analysis def
\newcommand{\tauh}{\ensuremath{\tau_\mathrm{h}}\xspace}

\newcommand{\ttbar}{\ensuremath{{\rm t\bar{t}}}\xspace}
\newcommand{\ttbarZ}{\ensuremath{{\ttbar{}\mathrm{Z}}}\xspace}
\newcommand{\ttbarW}{\ensuremath{{\ttbar{}\mathrm{W}}}\xspace}


\newcommand{\ntops}{\ensuremath{N_{\text{tops}}}\xspace}
\newcommand{\nbjets}{\ensuremath{N_{\text{b-jets}}}\xspace}
\newcommand{\MET}{\ensuremath{p^{\rm miss}_{\rm T}}\xspace}
\newcommand{\MTTwo}{\ensuremath{M_{\mathrm{T2}}}\xspace}
\newcommand{\HT}{\ensuremath{H_{\mathrm{T}}}\xspace}

\newcommand{\MT}{\ensuremath{M_{\mathrm{T}}}\xspace}


\chapter{Introduction}
Particle physics is a branch of the physics that studies the nature of the particles that constitute matter and radiation. We have a glory history of discoveries in the particle physics in the past 50 years. The standard model is a beautiful assembles of these remarkable breakthroughs. In 2012, July 4th, the discovery of the Higgs boson provides us the last piece of the puzzle.

However, the standard model is not perfect. The hierarchy issue arises with the "small" mass of the Higgs particle. The absence of the explanation in the standard model for the cold dark matter is also a blemish on the remarkable artifact.

The supersymmetric extension of the standard model is a promising solution on these issues. The hierarchy problem in the standard model can be cured with the existence of the TeV level lightest supersymmetric particle (LSP). The LSP can be neutral and stable under R-parity conservation, which provide us a best candidate on the dark matter. Therefore, the supersymmetry theory becomes one exciting theory for us to explore.

The CERN Large Hadron Collider (LHC) was designed and build with the goal of the Higgs and new physics discoveries. The LHC is designed with the highest center-of-mass energy on 14 TeV, which is the summit of the energy that human designed expriment can reach so far. This high-energy high-intensity beast provides us the opportunity to reveal the mask of new physics. The compact muon solenoid (CMS) is a general-purpose detection system installed on the collision point 5 on the LHC. The high-resolution data is taken with high efficiency in the operation. All analysis results on the CMS collaboration is based on the LHC-CMS dataset, including my analysis, targeting on the supersymmetry search in all hadronic final state.

The results of the search in missing transverse energy, extended transverse mass, b-jets and top-jets final state are demonstrated in this paper. The backgrounds are carefully estimated with robust methods in this all-hadronic final state search. In addition, our analysis group designs a customized top-jet tagger algorithm in order to obtain the relatively high efficiency in all case.

The organization of this thesis is described following. The chapter 2 gives an overview of the theory of the standard model and its supersymmetric extension, with a detailed discussion on simplified models. A brief summary of the current status of the searches for supersymmetry is given at the end of the chapter. The LHC machine and CMS experiment are described in chapter 3. The method of the analysis and the physics interpretations are demonstrated in chapter 4. Finally, the thesis is concluded in Chapter 5.
 %usually intro
\chapter{The Standard Model and The Supersymmetry Theory}

\section{The Standard Model}
\subsection{1}
\subsection{2}
\subsection{3}

\section{The Supersymmetric Extention of Standard Model}
\subsection{1}
\subsection{2}
\subsection{3}

\chapter{The LHC and The CMS Experiment}

\section{The LHC}
The Large Hadron Collider(LHC) is the largest and most powerful superconducting hadron accelerator and collider. The LHC was installed in an existing 26.7 km tunnel that was constructed for LEP machine between 1984 and 1989. The tunnel of LHC has 8 straight sections and 8 arcs, and locates between 45m and 170m below the surface. The LHC host 4 experiments currently: CMS(Point 5), ATLAS(Point 1), ALICE(Point 2) and LCHb(Point 8).
\subsection{LHC: Performance and Main machine layout}

\begin{figure}[htbp]
 \begin{center}
  \includegraphics[width=0.8\textwidth]{figures/c3/c3_lhc_latticelayout.jpg}
 \end{center}
 \caption{abcf}
 \label{fig:c3lhclayout}
\end{figure}
Performance, eq, lumi, xsection. Lumi equation from lhc parameter.
Main machine layout: 2 beam with 2 rings, 8 long straight sections and 8 arcs.

\begin{equation}
 L = \frac{N^{2}_{b}n_{b}f_{rev}\gamma_{r}}{4\pi \varepsilon_{n}\beta *}F \;
 \label{eq:c3lhclumi}
\end{equation}

\begin{equation}
 F = (1+\frac{\theta_{c}\sigma_{c}}{2\sigma *})^{-1/2} \;
 \label{eq:c3lhcgeof}
\end{equation}

\subsection{LHC: From operation point of view}
The LHC is a extremely complex machine and it is almost impossible to grasp all details. However, the LHC is provide a summary of status for the operational activities, which are very useful in the detector operation.
\subsubsection{Acclerator Mode}
The accelerator mode provides a summary status of the LHC machine.The detector system(eg. CMS) need to make daily operational decision accroding to the accelerator mode.
\begin{table}[htbp]
\fontsize{10 pt}{1.2 em}
\selectfont
\begin{centering}
\caption{\label{tab:c3lhcaccmode} Acclerator Mode}
\hspace*{-4ex}
\begin{tabular}{|c|c|c|}
\hline
 Mode Name &  Description & Beam exist \\
\hline
 SHUTDOWN & \specialcell{Machine not running} & NO BEAM \\
\hline
 COOLDOWN & \specialcell{Machine comes back from shutdown,\\ cryogenics related activities going on} & NO BEAM \\
\hline
 MACHINE CHECKOUT & \specialcell{Checking out LHC subsystems} & NO BEAM \\
\hline
 ACCESS & \specialcell{Access going on} & NO BEAM \\
\hline
 MACHINE TEST & \specialcell{Operation tests without beam} & NO BEAM \\
\hline
 CALIBRATION & \specialcell{Power converter calibration} & NO BEAM \\
\hline
 WARM-UP & \specialcell{Sectors warm up for repair} & NO BEAM \\
\hline
 RECOVERY & \specialcell{Quench recovery} & NO BEAM \\
\hline
 SECTOR DEPENDENT & \specialcell{Sector activities going on} & NO BEAM \\
\hline
 BEAM SETUP & \specialcell{Machine setup with 1 or 2 beams,\\ usually a signal of next physics fill when taking data} & BEAM \\
\hline
 PROTON PHYSICS & \specialcell{Beam on for proton physics} & BEAM \\
\hline
 ION PHYSICS & \specialcell{Beam on for ion physics} & BEAM \\
\hline
 TOTEM PHYSICS & \specialcell{Beam on for TOTEM physics} & BEAM \\
\hline
 MACHINE DEVELOPMETN & \specialcell{Beam on machine development} & BEAM \\
\hline
\end{tabular}
\par\end{centering}
\end{table}

\subsubsection{Beam Mode}

\begin{table}[htbp]
\fontsize{10 pt}{1.2 em}
\selectfont
\begin{centering}
\caption{\label{tab:c3lhcbeammode} Beam Mode}
\hspace*{-4ex}
\begin{tabular}{|c|c|c|}
\hline
 Mode Name &  Description \\
\hline
 SETUP & \specialcell{Beam in transferline, but not in the ring} \\
\hline
 ABORT & \specialcell{Recovery mode following bram drop} \\
\hline
 INJECTION PROBE BEAM & \specialcell{Ring is injected with test beam for safe circulating} \\
\hline
 INJECTION SETUP BEAM & \specialcell{Beam measurement going on after probe beam\\ but before injection physics beam} \\
\hline
 INJECTION PHYSICS BEAM & \specialcell{Beam for physics is injected in the ring} \\
\hline
 PRERAMP & \specialcell{Injection done, prepare for ramp} \\
\hline
 RAMP & \specialcell{Ramp up the beam energy} \\
\hline
 FLAT TOP & \specialcell{Ramp done, pre-squeeze checks} \\
\hline
 SQUEEZE & \specialcell{Squeezing the beam size} \\
\hline
 ADJUST & \specialcell{Preparing for collision after collision} \\
\hline
 STABLE BEAMS & \specialcell{Stable collision, detector should taking data} \\
\hline
 UNSTABLE BEAMS & \specialcell{Unstable beam because of sudden beam degradation} \\
\hline
 BEAM DUMP WARNING & \specialcell{Beam dump warning in case of emergency beam dump} \\
\hline
 BEAM DUMP & \specialcell{End of physics collision} \\
\hline
 RAMP DOWN & \specialcell{Ramp down beam energy after programmed dump} \\
\hline
 CYCLING & \specialcell{Pre-cycle before injection\\ following access, recovery, etc} \\
\hline
 NO BEAM  & \specialcell{No beam exist} \\
\hline
\end{tabular}
\par\end{centering}
\end{table}

\subsubsection{Operation Mode}

\section{The CMS Experiment}

The CMS Experiment is a particle physics experiment based on CMS detector system on the LHC. It contains with CMS detector system and event reconstruction, supported by the detector operation team, computing/storage department and software fraction.

\subsection{CMS Detector System}

The Compact Muon Solenoid (CMS) is one of the general-purpose detection system on the LHC. To fullfill the "general-purpose", the CMS is designed as a combo of several subsystems: Sillicon Pixels and Strips for tracking information, Electromagnetic and Hadron Calorimeters for "light" particle energy deposition and drift tubes and cathode strip/resisstive plate chambers for muon details. The locations of these subdectors in the CMS are not random: The sillicon tracking subsystems are located on the inner CMS, closest to the collision point, and the calorimeter subsystems after the sillicon subdetectors, because of its destructive detection nature; Finally the muon system in the outer layer to capture our heavy object which go through the calorimeter subsystems. To obtain the high performance, the CMS is immersed in a 4-T field, which is powered by a superconducting solenoid("S" in CMS). The magnet components have a major contribution on the CMS total weight: 12500 tones out of 14000. However, comparing with its heavy weight, the size of the CMC system is relatively small: only 5000 $m^{3}$. Then the given name "Compact" is imposed in front of "Muon Solenoid" since its density near to 3000 $kg/m^{3}$.

\subsubsection{Inner Tracking system}

As described in the name - - - inner tracking system, there are two main features for this CMS sub-detector: the closest sub-detector to the LHC beam and tracks reconstruction. The main challenge is a consequence of the first feature: the inner tracking systems have to be radiation hard during the expected lifetime of 10 years (LHC run1+run2+run3). More over, since the LHC have high intensity and small bunch crossing (25ns), we also need the inner tracking systems to have a fine granularity and fast response. The silicon based technology detector is the best option to satisfy all these requirements. On the other hand, the second feature actually comes from the physics requirement. We need to reconstruct the tracks of charged particles and secondary vertices in the event from the three dimensional hits. The tracks information are not only used in the charged particle reconstruction, but also applied in the particle flow algorithm as a base of all physics objects in CMS collaboration. The secondary vertices are also important in the heavy flavor objects reconstruction, and new physics search of long-lived particles. 

As a result of budget-performance balance, the CMS inner tracking systems are designed as a combination of two sub-systems: the silicon pixel and the silicon strip. A schematic drawing of the CMS tracking system is shown in the Fig.~\ref{fig:c3cms2dtracker}. More details will be described in the following sections. 

\begin{figure}[htbp]
 \begin{center}
  \includegraphics[width=0.8\textwidth]{figures/c3/c3_cms_2dtracker.png}
 \end{center}
 \caption{Two dimensional CMS inner tracking system layout, phrase 0}
 \label{fig:c3cms2dtracker}
\end{figure}

\paragraph{Silicon Pixels}

The pixel system is the part of inner tracking system that closest to the collision point. It provides a precise measurement of the tracking points; therefore contribute a lot for the secondary vertex reconstruction. The pixel cells are distributed with size of 100 num*150 num in three-dimensional space, which allow a 3D reconstruction method for the secondary vertex offline. Readout chips bump bonded to the sensor read out the signals from pixel cells. Then the signal will be delivered to the frondend driver through the analog chain once we receive a positive bit from L1 trigger.

The phrase 0 pixel detector contains three barrel layers (BPix) and two endcap disks (FPix), which covers a pseudo rapidity range $-2.5<\eta<2.5$. In phrase 1 upgrade, to suffer from the radiation damage, we replaced all layers and disks with new ones, and added one layer in the barrel region, one disk in the endcap region.

One special design of the pixel detector is the blade that module mounted are rotated by 20 degrees in a turbine geometry. Such an arrangement enhanced the charge sharing effects between the nearby pixels. The space resolution can be improved to 15num by the using the “charge sharing” effect in pixels. CMS take advantage of this effect with the analog readout of the collected charges by readout pixels in-group with readout chips. 

\paragraph{Silicon Strips}

The particles pass through ten layers of silicon strip detectors after the pixel. The strip detector provides a complete charged track together with the pixel detector. The silicon strip tracker is composed of 15148 detector modules. Each module carries either one 320 num or two 500 num silicon sensors from a total of 24244 sensors. The charges that come out of sensor will be amplified, shaped and stored by a custom integrated circuit. Once received a positive L1 trigger decision, the analog signal will be multiplexed and transmitted via optical link to the front end driver, then digitalized. 

The inner strip detector contains 4 barrel layers (TIB) and 3 disks (TID) support for each side. The outer strip detector has 6 barrel layers (TOB, 2 double-sided, 4 single-sided) and the endcap strip has 9 disks (TEC) on each side.

The online alignment system for the strip detector is necessary with the complicate structure and resolution requirement in physics performance. The laser alignment system is designed and built in to monitor the stability and the alignment of the strip detector mechanical structure. The infrared laser light will be delivered directly to the sensor on the 434 silicon modules (3\%) to trigger a signal pulse. The alignment data can be taken during commission, inter-fill period and also in the orbit gap when we have stable beam. Since the pixel will be aligned with strip through the reconstructed tracks, the strip alignment is the only online and direct alignment for the whole inner tracking system. 


\subsubsection{Calorimeters}
\paragraph{Electromagnetic Calorimeter}
\paragraph{Hadron Calorimeter}

\subsubsection{Muon Detectors}

The importance of the Muon detectors is implied by the experiment’s middle name (“M” is CMS). The CMS muon detectors provide precise measurement of muon tracks with the support of strong magnet field (3.8T). The measurement of muons is useful in both standard model physics (e.g. Higgs to ZZ) and new physics search (e.g. leptonic channel supersymmetry). More over, the muon system also provide muon related trigger to reduce the data rate.

The CMS muon detectors consist of three sub-detectors: Drift tube in the barrel region, cathode strip chamber in the endcap and resistive plate chamber in both barrel and endcap. A layout of CMS muon detectors are showed in the Fig.~\ref{fig:c3cms2dmuondets}. More details are demonstrated in the following sections. 

\begin{figure}[htbp]
 \begin{center}
  \includegraphics[width=0.8\textwidth]{figures/c3/c3_cms_2dmuondets.png}
 \end{center}
 \caption{Two dimensional CMS inner tracking system layout, phrase 0}
 \label{fig:c3cms2dmuondets}
\end{figure}

\paragraph{Drift Tubes}
Drift tubes (DT) are the part of CMS muon detectors that measure the muon tracks in the barrel region. The basic unit of the drift tubes is the drift cell. A drift cell is a 42mm-wide tube contains a thin conductive wire at high positive voltage within a gas volume. The muon knocks electron off the atom of the gas when it travel through the tube. The muon tracks can be reconstructed by measuring the drift time for different cells. 

Like the HCAL outer, the Drift tubes contain 5 wheels along the z-axis, 4 stations for each wheel, labeled as MB1, MB2, MB3 and MB4. The three inner stations have 60 drift chambers each and the outer chamber has 70. A drift chamber consists of 2 or 3 super layers, each made of by 4 layers of drift cells staggered by half of cell. This honeycomb geometry gives an excellent time-tagging capability, with a time resolution of a few nanoseconds. The full DT provides the pseudo-rapidity coverage $0<|\eta|<1.2$.

\paragraph{Cathode Strip Chambers}
Cathode strip chamber (CSC) is one of the CMS muon detectors that located in the endcap region. CSC is also a type of wire chamber, but different from the DT, CSC measures the location (to be specific, phi coordinates) instead of drift time. The CSC consists arrays of positively charged anode wires crossed with negatively charged copper cathode strip within a gas volume. The muon will knock the electron off from the gas atom. Then the electron will move to the anode wire to create an avalanche. Positive ions also move to the cathode and trigger a charged pulse.

The CMS endcap muon system consists of 468 cathode strip chambers in the following arrangement: 72 ME1/1, 72 ME1/2, 72 ME1/3, 36 ME2/1, 72 ME2/2, 36 ME3/1, 72 ME3/2, and 36 ME4/1. The full CSC provides the pseudo-rapidity coverage $1.2<|\eta|<2.4$. 

As we will mention at the beginning of next section, the RPC is designed as a fast response provider to trigger. However, for the endcap region, CSC is already good enough to satisfy the trigger requirement in the current instantaneous luminosity of the LHC. Moreover, CSC has a better spatial resolution and more pseudo-rapidity coverage, which provide a precise measurement for the endcap muons.

\paragraph{Resistive Plate Chambers}
Resistive plate chamber (RPC) is a fast gaseous detector that provides a muon trigger system in parallel with the DT and CSC. The RPC is a two high resistively plastic parallel plates system with gas in the middle. One of the plates is the positively charged anode and the other is negatively charged cathode. Like the CSC, the electron of gas will be knocked off when muon pass through, and then trigger avalanche. The pattern of hits on the cathodes will provide a quick measurement on the muon momentum and then pass to trigger for decision-making. In the CMS RPC, we are using the double-gap module instead of the single-gap one. This allows a lower bias voltage requirement of each single-gap and higher detector efficiency. The RPC performance is a combination of good spatial resolution and time resolution (one nanosecond). 

The CMS RPCs are distributed in both barrel and endcap region. There are 96 RPCs each wheel in barrel region. More details of the distribution are listed in the Table~\ref{tab:c3cmsrpc}. In the endcap region, we have 3 RPC stations in the phrase 0 design. In order to keep the high muon reconstruction efficiency with run2 condition, the fourth disk is installed during the long shutdown 1 for the phrase 1 upgrade. The full RPC have the same pseudo-rapidity coverage as DT in barrel, and smaller coverage ($1.2<|\eta|<2.1$) than CSC in endcap.

\begin{table}[htbp]
\fontsize{10 pt}{1.2 em}
\selectfont
\begin{centering}
\caption{\label{tab:c3cmsrpc} Numbers of RPCs for different wheels}
\hspace*{-4ex}
\begin{tabular}{|c|c|c|c|c|c|c|}
\hline
 RPC &  W+2 & W+1 & W0 & W-1 & W-2 & Total \\
\hline
 RB1(in) & 12 & 12 & 12 & 12 & 12 & 60 \\
\hline
 RB1(out) & 12 & 12 & 12 & 12 & 12 & 60 \\
\hline
 RB2/2(in) & 12 & - & - & - & 12 & 24 \\
\hline
 RB2/2(out) & - & 12 & 12 & 12 & - & 36 \\
\hline
 RB2/3(in) & - & 12 & 12 & 12 & - & 36 \\
\hline
 RB2/3(out) & 12 & - & - & - & 12 & 24 \\
\hline
 RB3 & 24 & 24 & 24 & 24 & 24 & 120 \\
\hline
 RB4 & 24 & 24 & 24 & 24 & 24 & 120 \\
\hline
 Total & 96 & 96 & 96 & 96 & 96 & 480 \\
\hline
\end{tabular}
\par\end{centering}
\end{table}


\subsubsection{Trigger system}


\subsection{Event Reconstrunction}
\subsubsection{Particle Flow}
\subsubsection{Tracks}
\subsubsection{Electrons}
\subsubsection{Muons}
\subsubsection{Jets and Missing Transverse Energy}

\chapter{SUSY Search with a customized top tagger}

\section{Customized top tagger}

\section{Event selection and Search bin design}

\section{Background Estimation}

\subsection{Backgrounds from top and W decays}
\begin{figure}[htbp]
 \begin{center}
  \includegraphics[width=0.8\textwidth]{figures/c4/c4_top_w_decaymod.png}
 \end{center}
 \caption{abcf}
 \label{fig:c4twdecaymod}
\end{figure}

\subsubsection{Classical LostLepton method}
\subsubsection{Classical Hadronic Tau method}
\subsubsection{Combined translation factor method}

\clearpage
\subsection{Backgrounds from neutrinos in Z decays}

\clearpage
\subsection{Backgrounds from QCD multi-jet}
\label{sec:c4bgqcd}
\input{sections/mc4/Backgrounds/QCD/qcd.tex}

\subsection{Backgrounds from TTZ and Other Standard Model Rare process}

\section{Interpretations}

\chapter{Conclusions}

A search for top squarks and gluinos in all-hadronic events produced in proton-proton collisions at center-of-mass energy 13 TeV has been presented. The data sample, collected in 2016 with the CMS detector at the CERN large hadron collider (LHC), corresponds to an integrated luminosity of 35.9 $fb^{-1}$.

The standard model backgrounds are estimated using data-driven methods as well as MC simulation. No excess of events above the expected standard model background is observed. The result is interpreted in the context of simplified supersymmetric models as 95\% confidence level upper limits on the cross section of gluinos and top squarks pair production processes. The T2tt model has been excluded for top squark masses up to 1020 GeV. The corresponding exclusions on the gluino mass are up to 1810-2040 GeV, depending on the type of models. The naturalness of the SUSY MSSM RPC simplified models are under challenge with large 13 TeV data samples now being collected at the LHC.


\nocite{*}
% \singlespacing
% \bibliographystyle{alpha}
\bibliographystyle{plain}
\bibliography{reference}

\renewcommand{\theHchapter}{A\arabic{chapter}}
\appendix
\addtocontents{toc}{\protect\setcounter{tocdepth}{3}} 
\chapter{Supplementary material for analysis}
\clearpage
\section{Quark-gluon discriminator study}

The $p_{t}D$ in terms of jet $\eta$ in different jet flavors is showed in Fig~\ref{fig:c4ttqgptdjeteta}.
\begin{figure}[htbp]
 \begin{center}
  \includegraphics[width=0.45\textwidth]{sections/mc4/TopTagger/figures/_b_qgptdjetetabin0_.png}
  \includegraphics[width=0.45\textwidth]{sections/mc4/TopTagger/figures/_b_qgptdjetetabin1_.png} \\
  \includegraphics[width=0.45\textwidth]{sections/mc4/TopTagger/figures/_b_qgptdjetetabin2_.png}
  \includegraphics[width=0.45\textwidth]{sections/mc4/TopTagger/figures/_b_qgptdjetetabin3_.png} \\
  \includegraphics[width=0.45\textwidth]{sections/mc4/TopTagger/figures/_b_qgptdjetetabin4_.png}
  \includegraphics[width=0.45\textwidth]{sections/mc4/TopTagger/figures/_b_qgptdjetetabin5_.png}
 \end{center}
 \caption{Top left: Quark Gluon $p_{t}D$ for jet $\eta$ bin 1; Top right: jet $\eta$ bin 2; Middle left: jet $\eta$ bin 3; Middle right: jet $\eta$ bin 4; Middle left: jet $\eta$ bin 5; Middle right: jet $\eta$ bin 6}
 \label{fig:c4ttqgptdjeteta}
\end{figure}

The $p_{t}D$ in terms of jet $p_{T}$ in different jet flavors is showed in Fig~\ref{fig:c4ttqgptdjetpt}.
\begin{figure}[htbp]
 \begin{center}
  \includegraphics[width=0.45\textwidth]{sections/mc4/TopTagger/figures/_b_qgptdjetptbin0_.png}
  \includegraphics[width=0.45\textwidth]{sections/mc4/TopTagger/figures/_b_qgptdjetptbin1_.png} \\
  \includegraphics[width=0.45\textwidth]{sections/mc4/TopTagger/figures/_b_qgptdjetptbin2_.png}
  \includegraphics[width=0.45\textwidth]{sections/mc4/TopTagger/figures/_b_qgptdjetptbin3_.png} \\
  \includegraphics[width=0.45\textwidth]{sections/mc4/TopTagger/figures/_b_qgptdjetptbin4_.png}
  \includegraphics[width=0.45\textwidth]{sections/mc4/TopTagger/figures/_b_qgptdjetptbin5_.png}
 \end{center}
 \caption{Top left: Quark Gluon $p_{t}D$ for jet $p_{T}$ bin 1; Top right: jet $p_{T}$ bin 2; Middle left: jet $p_{T}$ bin 3; Middle right: jet $p_{T}$ bin 4; Middle left: jet $p_{T}$ bin 5; Middle right: jet $p_{T}$ bin 6}
 \label{fig:c4ttqgptdjetpt}
\end{figure}

The multiplicity in terms of jet $\eta$ in different jet flavors is showed in Fig~\ref{fig:c4ttqgmultjeteta}.
\begin{figure}[htbp]
 \begin{center}
  \includegraphics[width=0.45\textwidth]{sections/mc4/TopTagger/figures/_b_qgmultjetetabin0_.png}
  \includegraphics[width=0.45\textwidth]{sections/mc4/TopTagger/figures/_b_qgmultjetetabin1_.png} \\
  \includegraphics[width=0.45\textwidth]{sections/mc4/TopTagger/figures/_b_qgmultjetetabin2_.png}
  \includegraphics[width=0.45\textwidth]{sections/mc4/TopTagger/figures/_b_qgmultjetetabin3_.png} \\
  \includegraphics[width=0.45\textwidth]{sections/mc4/TopTagger/figures/_b_qgmultjetetabin4_.png}
  \includegraphics[width=0.45\textwidth]{sections/mc4/TopTagger/figures/_b_qgmultjetetabin5_.png}
 \end{center}
 \caption{Top left: Quark Gluon multiplicity for jet $\eta$ bin 1; Top right: jet $\eta$ bin 2; Middle left: jet $\eta$ bin 3; Middle right: jet $\eta$ bin 4; Middle left: jet $\eta$ bin 5; Middle right: jet $\eta$ bin 6}
 \label{fig:c4ttqgmultjeteta}
\end{figure}

The multiplicity in terms of jet $p_{T}$ in different jet flavors is showed in Fig~\ref{fig:c4ttqgmultjetpt}.
\begin{figure}[htbp]
 \begin{center}
  \includegraphics[width=0.45\textwidth]{sections/mc4/TopTagger/figures/_b_qgmultjetptbin0_.png}
  \includegraphics[width=0.45\textwidth]{sections/mc4/TopTagger/figures/_b_qgmultjetptbin1_.png} \\
  \includegraphics[width=0.45\textwidth]{sections/mc4/TopTagger/figures/_b_qgmultjetptbin2_.png}
  \includegraphics[width=0.45\textwidth]{sections/mc4/TopTagger/figures/_b_qgmultjetptbin3_.png} \\
  \includegraphics[width=0.45\textwidth]{sections/mc4/TopTagger/figures/_b_qgmultjetptbin4_.png}
  \includegraphics[width=0.45\textwidth]{sections/mc4/TopTagger/figures/_b_qgmultjetptbin5_.png}
 \end{center}
 \caption{Top left: Quark Gluon multiplicity for jet $p_{T}$ bin 1; Top right: jet $p_{T}$ bin 2; Middle left: jet $p_{T}$ bin 3; Middle right: jet $p_{T}$ bin 4; Middle left: jet $p_{T}$ bin 5; Middle right: jet $p_{T}$ bin 6}
 \label{fig:c4ttqgmultjetpt}
\end{figure}

The Axis2 in terms of jet $\eta$ in different jet flavors is showed in Fig~\ref{fig:c4ttqgaxis2jeteta}.
\begin{figure}[htbp]
 \begin{center}
  \includegraphics[width=0.45\textwidth]{sections/mc4/TopTagger/figures/_b_qgaxis2jetetabin0_.png}
  \includegraphics[width=0.45\textwidth]{sections/mc4/TopTagger/figures/_b_qgaxis2jetetabin1_.png} \\
  \includegraphics[width=0.45\textwidth]{sections/mc4/TopTagger/figures/_b_qgaxis2jetetabin2_.png}
  \includegraphics[width=0.45\textwidth]{sections/mc4/TopTagger/figures/_b_qgaxis2jetetabin3_.png} \\
  \includegraphics[width=0.45\textwidth]{sections/mc4/TopTagger/figures/_b_qgaxis2jetetabin4_.png}
  \includegraphics[width=0.45\textwidth]{sections/mc4/TopTagger/figures/_b_qgaxis2jetetabin5_.png}
 \end{center}
 \caption{Top left: Quark Gluon Axis2 for jet $\eta$ bin 1; Top right: jet $\eta$ bin 2; Middle left: jet $\eta$ bin 3; Middle right: jet $\eta$ bin 4; Middle left: jet $\eta$ bin 5; Middle right: jet $\eta$ bin 6}
 \label{fig:c4ttqgaxis2jeteta}
\end{figure}

The Axis2 in terms of jet $p_{T}$ in different jet flavors is showed in Fig~\ref{fig:c4ttqgaxis2jetpt}.
\begin{figure}[htbp]
 \begin{center}
  \includegraphics[width=0.45\textwidth]{sections/mc4/TopTagger/figures/_b_qgaxis2jetptbin0_.png}
  \includegraphics[width=0.45\textwidth]{sections/mc4/TopTagger/figures/_b_qgaxis2jetptbin1_.png} \\
  \includegraphics[width=0.45\textwidth]{sections/mc4/TopTagger/figures/_b_qgaxis2jetptbin2_.png}
  \includegraphics[width=0.45\textwidth]{sections/mc4/TopTagger/figures/_b_qgaxis2jetptbin3_.png} \\
  \includegraphics[width=0.45\textwidth]{sections/mc4/TopTagger/figures/_b_qgaxis2jetptbin4_.png}
  \includegraphics[width=0.45\textwidth]{sections/mc4/TopTagger/figures/_b_qgaxis2jetptbin5_.png}
 \end{center}
 \caption{Top left: Quark Gluon Axis2 for jet $p_{T}$ bin 1; Top right: jet $p_{T}$ bin 2; Middle left: jet $p_{T}$ bin 3; Middle right: jet $p_{T}$ bin 4; Middle left: jet $p_{T}$ bin 5; Middle right: jet $p_{T}$ bin 6}
 \label{fig:c4ttqgaxis2jetpt}
\end{figure}

\clearpage
\section{Di-Top jets event display}
The Event with two reconstructed top jets are demonstrated in Fig~\ref{fig:appttevtdisplay}. The jet energies and \MET shown are without energy correction.
\begin{figure}[htbp]
 \begin{center}
  \includegraphics[width=0.45\textwidth]{figures/appendix/appendix_tagger_Run2016B_2t_1j3j-274999_918496948_499_RhoPhi.png}
  \includegraphics[width=0.45\textwidth]{figures/appendix/appendix_tagger_Run2016B_2t_2j3j-274250_425322875_212_RhoPhi.png}
\end{center} \caption{Event display for di-topjets event in data. Left plot is one mono-jet top and one triple-jet top, Right plot is one di-jet top and one triple-jet top. Yellow line is fat jet (anti-$k_{T}$ R=0.8), blue line is normal jet (anti-$k_{T}$ R=0.4), the purple line is \MET.}
 \label{fig:appttevtdisplay}
\end{figure}

\clearpage
\section{Aggregate search bin, simplified top tagger and results}

\chapter{Service work}
\clearpage

\section{HCAL logical map}
HCAL Logical map has been described in detail in the HCAL section in chapter 3. The HCAL map validation plots are listed here in different period and various sub-detectors. The FrontEnd electronics variables are the based coordinates that used in the validation plots. The validation plots are separated into 2 categories: BackEnd variables in FrontEnd coordinates and Geometry variables in FrontEnd coordinates. A complete FrontEnd coordinates set has 4 variables, RBX (readout box), RM (readout module), RM fibers and fiber channel for HB, HE and HO. HF is special since it does have RM (readout module). HF FrontEnd can be described by RBX, QIE10 slot, QIE10 fiber and fiber channel. BackEnd and Geometry variables will be described separately in the following sections.
\clearpage

\subsection{Remapped phase 1 HB in 2018}
The 2018 HB is still in HPD+QIE8 FrontEnd+uTCA BackEnd stage. We are in this hardware setting since 2016. However, HE will be upgraded to SiPM+QIE11 FrontEnd+uTCA BackEnd with more readout channels. Then the patch panels need to be reorganized since HB and HE share the same BackEnd electronics. The remapped 2018 HB is followed by the design from Richard Kellogg, coordinated with Jeremy Mans on the requirement of trigger sum algorithm in BackEnd firmware. 

The validation plots of remapped HB in 2018 are showed from Fig~\ref{fig:lmapHBPEtaFEC} to Fig~\ref{fig:lmapHBMuHTRFIFEC}. There are 2592 readout channels in total for remapped HB (same as 2016 HB, before remap). In order to satisfy the trigger latency requirement, the frontend channel permutation is applied in this map. HB BackEnd electronics are still in uTCA, 12 uHTR in one crate, 24 out going fibers per uHTR card, with 3 channels per fiber. 
\clearpage

\begin{figure}[htb]
 \begin{center}
  \begin{tabular}{cc}
   \includegraphics[angle=0,width=0.95\textwidth]{figures/appendix/HBP_Eta_in_FrontEnd.png}
  \end{tabular}
	\caption{HCAL (phase 1 HB, plus side) detector $\eta$ distribution in the frontend electronic coordinates.}
  \label{fig:lmapHBPEtaFEC}
 \end{center}
\end{figure}
\clearpage

\begin{figure}[htb]
 \begin{center}
  \begin{tabular}{cc}
   \includegraphics[angle=0,width=0.95\textwidth]{figures/appendix/HBP_Phi_in_FrontEnd.png}
  \end{tabular}
	\caption{HCAL (phase 1 HB, plus side) detector $\phi$ distribution in the frontend electronic coordinates.}
  \label{fig:lmapHBPPhiFEC}
 \end{center}
\end{figure}
\clearpage

\begin{figure}[htb]
 \begin{center}
  \begin{tabular}{cc}
   \includegraphics[angle=0,width=0.95\textwidth]{figures/appendix/HBP_Depth_in_FrontEnd.png}
  \end{tabular}
	\caption{HCAL (phase 1 HB, plus side) detector depth distribution in the frontend electronic coordinates.}
  \label{fig:lmapHBPDepthFEC}
 \end{center}
\end{figure}
\clearpage

\begin{figure}[htb]
 \begin{center}
  \begin{tabular}{cc}
   \includegraphics[angle=0,width=0.95\textwidth]{figures/appendix/HBM_Eta_in_FrontEnd.png}
  \end{tabular}
  \caption{HCAL (phase 1 HB, minus side) detector $\eta$ distribution in the frontend electronic coordinates.}
  \label{fig:lmapHBMEtaFEC}
 \end{center}
\end{figure}
\clearpage

\begin{figure}[htb]
 \begin{center}
  \begin{tabular}{cc}
   \includegraphics[angle=0,width=0.95\textwidth]{figures/appendix/HBM_Phi_in_FrontEnd.png}
  \end{tabular}
  \caption{HCAL (phase 1 HB, minus side) detector $\phi$ distribution in the frontend electronic coordinates.}
  \label{fig:lmapHBMPhiFEC}
 \end{center}
\end{figure}
\clearpage

\begin{figure}[htb]
 \begin{center}
  \begin{tabular}{cc}
   \includegraphics[angle=0,width=0.95\textwidth]{figures/appendix/HBM_Depth_in_FrontEnd.png}
  \end{tabular}
  \caption{HCAL (phase 1 HB, minus side) detector depth distribution in the frontend electronic coordinates.}
  \label{fig:lmapHBMDepthFEC}
 \end{center}
\end{figure}
\clearpage

\begin{figure}[htb]
 \begin{center}
  \begin{tabular}{cc}
   \includegraphics[angle=0,width=0.95\textwidth]{figures/appendix/HBP_Crate_in_FrontEnd.png}
  \end{tabular}
  \caption{HCAL (phase 1 HB, plus side) backend electronic coordinate crate distribution in the frontend electronic coordinates.}
  \label{fig:lmapHBPCrateFEC}
 \end{center}
\end{figure}
\clearpage

\begin{figure}[htb]
 \begin{center}
  \begin{tabular}{cc}
   \includegraphics[angle=0,width=0.95\textwidth]{figures/appendix/HBP_uHTR_in_FrontEnd.png}
  \end{tabular}
  \caption{HCAL (phase 1 HB, plus side) backend electronic coordinate uHTR slot distribution in the frontend electronic coordinates.}
  \label{fig:lmapHBPuHTRFEC}
 \end{center}
\end{figure}
\clearpage

\begin{figure}[htb]
 \begin{center}
  \begin{tabular}{cc}
   \includegraphics[angle=0,width=0.95\textwidth]{figures/appendix/HBP_uHTR_FI_in_FrontEnd.png}
  \end{tabular}
  \caption{HCAL (phase 1 HB, plus side) backend electronic coordinate uHTR fiber distribution in the frontend electronic coordinates.}
  \label{fig:lmapHBPuHTRFIFEC}
 \end{center}
\end{figure}
\clearpage

\begin{figure}[htb]
 \begin{center}
  \begin{tabular}{cc}
   \includegraphics[angle=0,width=0.95\textwidth]{figures/appendix/HBM_Crate_in_FrontEnd.png}
  \end{tabular}
  \caption{HCAL (phase 1 HB, minus side) backend electronic coordinate crate distribution in the frontend electronic coordinates.}
  \label{fig:lmapHBMCrateFEC}
 \end{center}
\end{figure}
\clearpage

\begin{figure}[htb]
 \begin{center}
  \begin{tabular}{cc}
   \includegraphics[angle=0,width=0.95\textwidth]{figures/appendix/HBM_uHTR_in_FrontEnd.png}
  \end{tabular}
  \caption{HCAL (phase 1 HB, minus side) backend electronic coordinate uHTR slot distribution in the frontend electronic coordinates.}
  \label{fig:lmapHBMuHTRFEC}
 \end{center}
\end{figure}
\clearpage

\begin{figure}[htb]
 \begin{center}
  \begin{tabular}{cc}
   \includegraphics[angle=0,width=0.95\textwidth]{figures/appendix/HBM_uHTR_FI_in_FrontEnd.png}
  \end{tabular}
  \caption{HCAL (phase 1 HB, minus side) backend electronic coordinate uHTR fiber distribution in the frontend electronic coordinates.}
  \label{fig:lmapHBMuHTRFIFEC}
 \end{center}
\end{figure}
\clearpage

\subsection{Phase 2 HB in 2020}
The 2018 HB will be replaced and move to SiPM+QIE11 FrontEnd+uTCA BackEnd stage in 2020, during the long shutdown 2. A complicate swap scheme is applied inside the FrontEnd board to satisfy the trigger latency requirement. Fortunately, thanks to Dick’s elegant design, we do not need to unplug HE fibers from patch panel to fit in the new 2020 HB.

The phase 2 2020 HB validation plots are shown from Fig~\ref{fig:lmapngHBPEtaFEC} to Fig~\ref{fig:lmapngHBMuHTRFIFEC}. There are 9216 readout channels. 9072 channels are physical readout channel, corresponding to actual tower and layer on detector. 144 of them are dummy calibration channels, which are readout by FrontEnd with no corresponding detector components. Those dummy calibration channels are labeled as "HBX" channels. HBX channels are distributed among 144 HB readout modules, but only on two locations: RM fiber 2 fiber channel 2 and RM fiber 3 fiber channel 6. BackEnd electronics are still in 12 uHTR and 24 uHTR fibers, but number of fiber channels per fiber is increased from 3 to 8, to accommodating more readout channels, comparing with 2018 remapped HB.
\clearpage

\begin{figure}[htb]
 \begin{center}
  \begin{tabular}{cc}
   \includegraphics[angle=0,width=0.95\textwidth]{figures/appendix/ngHBP_Eta_in_FrontEnd.png}
  \end{tabular}
	\caption{HCAL (phase 2 HB, plus side) detector $\eta$ distribution in the frontend electronic coordinates.}
  \label{fig:lmapngHBPEtaFEC}
 \end{center}
\end{figure}
\clearpage

\begin{figure}[htb]
 \begin{center}
  \begin{tabular}{cc}
   \includegraphics[angle=0,width=0.95\textwidth]{figures/appendix/ngHBP_Phi_in_FrontEnd.png}
  \end{tabular}
	\caption{HCAL (phase 2 HB, plus side) detector $\phi$ distribution in the frontend electronic coordinates.}
  \label{fig:lmapngHBPPhiFEC}
 \end{center}
\end{figure}
\clearpage

\begin{figure}[htb]
 \begin{center}
  \begin{tabular}{cc}
   \includegraphics[angle=0,width=0.95\textwidth]{figures/appendix/ngHBP_Depth_in_FrontEnd.png}
  \end{tabular}
	\caption{HCAL (phase 2 HB, plus side) detector depth distribution in the frontend electronic coordinates.}
  \label{fig:lmapngHBPDepthFEC}
 \end{center}
\end{figure}
\clearpage

\begin{figure}[htb]
 \begin{center}
  \begin{tabular}{cc}
   \includegraphics[angle=0,width=0.95\textwidth]{figures/appendix/ngHBM_Eta_in_FrontEnd.png}
  \end{tabular}
  \caption{HCAL (phase 2 HB, minus side) detector $\eta$ distribution in the frontend electronic coordinates.}
  \label{fig:lmapngHBMEtaFEC}
 \end{center}
\end{figure}
\clearpage

\begin{figure}[htb]
 \begin{center}
  \begin{tabular}{cc}
   \includegraphics[angle=0,width=0.95\textwidth]{figures/appendix/ngHBM_Phi_in_FrontEnd.png}
  \end{tabular}
  \caption{HCAL (phase 2 HB, minus side) detector $\phi$ distribution in the frontend electronic coordinates.}
  \label{fig:lmapngHBMPhiFEC}
 \end{center}
\end{figure}
\clearpage

\begin{figure}[htb]
 \begin{center}
  \begin{tabular}{cc}
   \includegraphics[angle=0,width=0.95\textwidth]{figures/appendix/ngHBM_Depth_in_FrontEnd.png}
  \end{tabular}
  \caption{HCAL (phase 2 HB, minus side) detector depth distribution in the frontend electronic coordinates.}
  \label{fig:lmapngHBMDepthFEC}
 \end{center}
\end{figure}
\clearpage

\begin{figure}[htb]
 \begin{center}
  \begin{tabular}{cc}
   \includegraphics[angle=0,width=0.95\textwidth]{figures/appendix/ngHBP_Crate_in_FrontEnd.png}
  \end{tabular}
  \caption{HCAL (phase 2 HB, plus side) backend electronic coordinate crate distribution in the frontend electronic coordinates.}
  \label{fig:lmapngHBPCrateFEC}
 \end{center}
\end{figure}
\clearpage

\begin{figure}[htb]
 \begin{center}
  \begin{tabular}{cc}
   \includegraphics[angle=0,width=0.95\textwidth]{figures/appendix/ngHBP_uHTR_in_FrontEnd.png}
  \end{tabular}
  \caption{HCAL (phase 2 HB, plus side) backend electronic coordinate uHTR slot distribution in the frontend electronic coordinates.}
  \label{fig:lmapngHBPuHTRFEC}
 \end{center}
\end{figure}
\clearpage

\begin{figure}[htb]
 \begin{center}
  \begin{tabular}{cc}
   \includegraphics[angle=0,width=0.95\textwidth]{figures/appendix/ngHBP_uHTR_FI_in_FrontEnd.png}
  \end{tabular}
  \caption{HCAL (phase 2 HB, plus side) backend electronic coordinate uHTR fiber distribution in the frontend electronic coordinates.}
  \label{fig:lmapngHBPuHTRFIFEC}
 \end{center}
\end{figure}
\clearpage

\begin{figure}[htb]
 \begin{center}
  \begin{tabular}{cc}
   \includegraphics[angle=0,width=0.95\textwidth]{figures/appendix/ngHBM_Crate_in_FrontEnd.png}
  \end{tabular}
  \caption{HCAL (phase 2 HB, minus side) backend electronic coordinate crate distribution in the frontend electronic coordinates.}
  \label{fig:lmapngHBMCrateFEC}
 \end{center}
\end{figure}
\clearpage

\begin{figure}[htb]
 \begin{center}
  \begin{tabular}{cc}
   \includegraphics[angle=0,width=0.95\textwidth]{figures/appendix/ngHBM_uHTR_in_FrontEnd.png}
  \end{tabular}
  \caption{HCAL (phase 2 HB, minus side) backend electronic coordinate uHTR slot distribution in the frontend electronic coordinates.}
  \label{fig:lmapngHBMuHTRFEC}
 \end{center}
\end{figure}
\clearpage

\begin{figure}[htb]
 \begin{center}
  \begin{tabular}{cc}
   \includegraphics[angle=0,width=0.95\textwidth]{figures/appendix/ngHBM_uHTR_FI_in_FrontEnd.png}
  \end{tabular}
  \caption{HCAL (phase 2 HB, minus side) backend electronic coordinate uHTR fiber distribution in the frontend electronic coordinates.}
  \label{fig:lmapngHBMuHTRFIFEC}
 \end{center}
\end{figure}
\clearpage

\subsection{Phase 1 HE in 2018}
The 2018 HE is in SiPM+QIE11 FrontEnd+uTCA BackEnd stage. HE mapping algorithm is always more complicate than HB due to the tricky geometry structure in detector coordinates. Comparing to 2016 HE with QIE8, no swap trick inside the FrontEnd board is applied in 2018 HE. This indeed increases the difficulty for the firmware expert to pin down the latency within L1 trigger requirement. 

The phase 1 2018 HE validation plots are shown from Fig~\ref{fig:lmapngHEPEtaFEC} to Fig~\ref{fig:lmapngHEMuHTRFIFEC}. There are 6912 readout channels. 6768 channels are physical readout channel, corresponding to actual tower and layer on detector. 144 of them are dummy calibration channels, which are readout by FrontEnd with no corresponding detector components. Those dummy calibration channels are labeled as “HEX” channels. HEX channels are distributed among 144 HE readout modules, on four different locations: RM fiber 2 fiber channel 1, RM fiber 3 fiber channel 6, RM fiber 5 fiber channel 5 and RM fiber 7 fiber channel 1. BackEnd electronics are still in 12 uHTR and 24 uHTR fibers, but number of fiber channels per fiber is increased from 3 to 6, to accommodating more readout channels, comparing with 2016 HE.
\clearpage

\begin{figure}[htb]
 \begin{center}
  \begin{tabular}{cc}
   \includegraphics[angle=0,width=0.95\textwidth]{figures/appendix/ngHEP_Eta_in_FrontEnd.png}
  \end{tabular}
	\caption{HCAL (phase 1 HE, plus side) detector $\eta$ distribution in the frontend electronic coordinates.}
  \label{fig:lmapngHEPEtaFEC}
 \end{center}
\end{figure}
\clearpage

\begin{figure}[htb]
 \begin{center}
  \begin{tabular}{cc}
   \includegraphics[angle=0,width=0.95\textwidth]{figures/appendix/ngHEP_Phi_in_FrontEnd.png}
  \end{tabular}
	\caption{HCAL (phase 1 HE, plus side) detector $\phi$ distribution in the frontend electronic coordinates.}
  \label{fig:lmapngHEPPhiFEC}
 \end{center}
\end{figure}
\clearpage

\begin{figure}[htb]
 \begin{center}
  \begin{tabular}{cc}
   \includegraphics[angle=0,width=0.95\textwidth]{figures/appendix/ngHEP_Depth_in_FrontEnd.png}
  \end{tabular}
	\caption{HCAL (phase 1 HE, plus side) detector depth distribution in the frontend electronic coordinates.}
  \label{fig:lmapngHEPDepthFEC}
 \end{center}
\end{figure}
\clearpage

\begin{figure}[htb]
 \begin{center}
  \begin{tabular}{cc}
   \includegraphics[angle=0,width=0.95\textwidth]{figures/appendix/ngHEM_Eta_in_FrontEnd.png}
  \end{tabular}
  \caption{HCAL (phase 1 HE, minus side) detector $\eta$ distribution in the frontend electronic coordinates.}
  \label{fig:lmapngHEMEtaFEC}
 \end{center}
\end{figure}
\clearpage

\begin{figure}[htb]
 \begin{center}
  \begin{tabular}{cc}
   \includegraphics[angle=0,width=0.95\textwidth]{figures/appendix/ngHEM_Phi_in_FrontEnd.png}
  \end{tabular}
  \caption{HCAL (phase 1 HE, minus side) detector $\phi$ distribution in the frontend electronic coordinates.}
  \label{fig:lmapngHEMPhiFEC}
 \end{center}
\end{figure}
\clearpage

\begin{figure}[htb]
 \begin{center}
  \begin{tabular}{cc}
   \includegraphics[angle=0,width=0.95\textwidth]{figures/appendix/ngHEM_Depth_in_FrontEnd.png}
  \end{tabular}
  \caption{HCAL (phase 1 HE, minus side) detector depth distribution in the frontend electronic coordinates.}
  \label{fig:lmapngHEMDepthFEC}
 \end{center}
\end{figure}
\clearpage

\begin{figure}[htb]
 \begin{center}
  \begin{tabular}{cc}
   \includegraphics[angle=0,width=0.95\textwidth]{figures/appendix/ngHEP_Crate_in_FrontEnd.png}
  \end{tabular}
  \caption{HCAL (phase 1 HE, plus side) backend electronic coordinate crate distribution in the frontend electronic coordinates.}
  \label{fig:lmapngHEPCrateFEC}
 \end{center}
\end{figure}
\clearpage

\begin{figure}[htb]
 \begin{center}
  \begin{tabular}{cc}
   \includegraphics[angle=0,width=0.95\textwidth]{figures/appendix/ngHEP_uHTR_in_FrontEnd.png}
  \end{tabular}
  \caption{HCAL (phase 1 HE, plus side) backend electronic coordinate uHTR slot distribution in the frontend electronic coordinates.}
  \label{fig:lmapngHEPuHTRFEC}
 \end{center}
\end{figure}
\clearpage

\begin{figure}[htb]
 \begin{center}
  \begin{tabular}{cc}
   \includegraphics[angle=0,width=0.95\textwidth]{figures/appendix/ngHEP_uHTR_FI_in_FrontEnd.png}
  \end{tabular}
  \caption{HCAL (phase 1 HE, plus side) backend electronic coordinate uHTR fiber distribution in the frontend electronic coordinates.}
  \label{fig:lmapngHEPuHTRFIFEC}
 \end{center}
\end{figure}
\clearpage

\begin{figure}[htb]
 \begin{center}
  \begin{tabular}{cc}
   \includegraphics[angle=0,width=0.95\textwidth]{figures/appendix/ngHEM_Crate_in_FrontEnd.png}
  \end{tabular}
  \caption{HCAL (phase 1 HE, minus side) backend electronic coordinate crate distribution in the frontend electronic coordinates.}
  \label{fig:lmapngHEMCrateFEC}
 \end{center}
\end{figure}
\clearpage

\begin{figure}[htb]
 \begin{center}
  \begin{tabular}{cc}
   \includegraphics[angle=0,width=0.95\textwidth]{figures/appendix/ngHEM_uHTR_in_FrontEnd.png}
  \end{tabular}
  \caption{HCAL (phase 1 HE, minus side) backend electronic coordinate uHTR slot distribution in the frontend electronic coordinates.}
  \label{fig:lmapngHEMuHTRFEC}
 \end{center}
\end{figure}
\clearpage

\begin{figure}[htb]
 \begin{center}
  \begin{tabular}{cc}
   \includegraphics[angle=0,width=0.95\textwidth]{figures/appendix/ngHEM_uHTR_FI_in_FrontEnd.png}
  \end{tabular}
  \caption{HCAL (phase 1 HE, minus side) backend electronic coordinate uHTR fiber distribution in the frontend electronic coordinates.}
  \label{fig:lmapngHEMuHTRFIFEC}
 \end{center}
\end{figure}
\clearpage

\subsection{Phase 1 HF in 2017}
\clearpage

\begin{figure}[htb]
 \begin{center}
  \begin{tabular}{cc}
   \includegraphics[angle=0,width=0.95\textwidth]{figures/appendix/ngHFP_Eta_in_FrontEnd.png}
  \end{tabular}
	\caption{HCAL (phase 1 HF, plus side) detector $\eta$ distribution in the frontend electronic coordinates.}
  \label{fig:lmapngHFPEtaFEC}
 \end{center}
\end{figure}
\clearpage

\begin{figure}[htb]
 \begin{center}
  \begin{tabular}{cc}
   \includegraphics[angle=0,width=0.95\textwidth]{figures/appendix/ngHFP_Phi_in_FrontEnd.png}
  \end{tabular}
	\caption{HCAL (phase 1 HF, plus side) detector $\phi$ distribution in the frontend electronic coordinates.}
  \label{fig:lmapngHFPPhiFEC}
 \end{center}
\end{figure}
\clearpage

\begin{figure}[htb]
 \begin{center}
  \begin{tabular}{cc}
   \includegraphics[angle=0,width=0.95\textwidth]{figures/appendix/ngHFP_Depth_in_FrontEnd.png}
  \end{tabular}
	\caption{HCAL (phase 1 HF, plus side) detector depth distribution in the frontend electronic coordinates.}
  \label{fig:lmapngHFPDepthFEC}
 \end{center}
\end{figure}
\clearpage

\begin{figure}[htb]
 \begin{center}
  \begin{tabular}{cc}
   \includegraphics[angle=0,width=0.95\textwidth]{figures/appendix/ngHFM_Eta_in_FrontEnd.png}
  \end{tabular}
  \caption{HCAL (phase 1 HF, minus side) detector $\eta$ distribution in the frontend electronic coordinates.}
  \label{fig:lmapngHFMEtaFEC}
 \end{center}
\end{figure}
\clearpage

\begin{figure}[htb]
 \begin{center}
  \begin{tabular}{cc}
   \includegraphics[angle=0,width=0.95\textwidth]{figures/appendix/ngHFM_Phi_in_FrontEnd.png}
  \end{tabular}
  \caption{HCAL (phase 1 HF, minus side) detector $\phi$ distribution in the frontend electronic coordinates.}
  \label{fig:lmapngHFMPhiFEC}
 \end{center}
\end{figure}
\clearpage

\begin{figure}[htb]
 \begin{center}
  \begin{tabular}{cc}
   \includegraphics[angle=0,width=0.95\textwidth]{figures/appendix/ngHFM_Depth_in_FrontEnd.png}
  \end{tabular}
  \caption{HCAL (phase 1 HF, minus side) detector depth distribution in the frontend electronic coordinates.}
  \label{fig:lmapngHFMDepthFEC}
 \end{center}
\end{figure}
\clearpage

\begin{figure}[htb]
 \begin{center}
  \begin{tabular}{cc}
   \includegraphics[angle=0,width=0.95\textwidth]{figures/appendix/ngHFP_Crate_in_FrontEnd.png}
  \end{tabular}
  \caption{HCAL (phase 1 HF, plus side) backend electronic coordinate crate distribution in the frontend electronic coordinates.}
  \label{fig:lmapngHFPCrateFEC}
 \end{center}
\end{figure}
\clearpage

\begin{figure}[htb]
 \begin{center}
  \begin{tabular}{cc}
   \includegraphics[angle=0,width=0.95\textwidth]{figures/appendix/ngHFP_uHTR_in_FrontEnd.png}
  \end{tabular}
  \caption{HCAL (phase 1 HF, plus side) backend electronic coordinate uHTR slot distribution in the frontend electronic coordinates.}
  \label{fig:lmapngHFPuHTRFEC}
 \end{center}
\end{figure}
\clearpage

\begin{figure}[htb]
 \begin{center}
  \begin{tabular}{cc}
   \includegraphics[angle=0,width=0.95\textwidth]{figures/appendix/ngHFP_uHTR_FI_in_FrontEnd.png}
  \end{tabular}
  \caption{HCAL (phase 1 HF, plus side) backend electronic coordinate uHTR fiber distribution in the frontend electronic coordinates.}
  \label{fig:lmapngHFPuHTRFIFEC}
 \end{center}
\end{figure}
\clearpage

\begin{figure}[htb]
 \begin{center}
  \begin{tabular}{cc}
   \includegraphics[angle=0,width=0.95\textwidth]{figures/appendix/ngHFM_Crate_in_FrontEnd.png}
  \end{tabular}
  \caption{HCAL (phase 1 HF, minus side) backend electronic coordinate crate distribution in the frontend electronic coordinates.}
  \label{fig:lmapngHFMCrateFEC}
 \end{center}
\end{figure}
\clearpage

\begin{figure}[htb]
 \begin{center}
  \begin{tabular}{cc}
   \includegraphics[angle=0,width=0.95\textwidth]{figures/appendix/ngHFM_uHTR_in_FrontEnd.png}
  \end{tabular}
  \caption{HCAL (phase 1 HF, minus side) backend electronic coordinate uHTR slot distribution in the frontend electronic coordinates.}
  \label{fig:lmapngHFMuHTRFEC}
 \end{center}
\end{figure}
\clearpage

\begin{figure}[htb]
 \begin{center}
  \begin{tabular}{cc}
   \includegraphics[angle=0,width=0.95\textwidth]{figures/appendix/ngHFM_uHTR_FI_in_FrontEnd.png}
  \end{tabular}
  \caption{HCAL (phase 1 HF, minus side) backend electronic coordinate uHTR fiber distribution in the frontend electronic coordinates.}
  \label{fig:lmapngHFMuHTRFIFEC}
 \end{center}
\end{figure}
\clearpage

\subsection{HO since 2015}
\clearpage

\begin{figure}[htb]
 \begin{center}
  \begin{tabular}{cc}
   \includegraphics[angle=0,width=0.95\textwidth]{figures/appendix/HO0_Eta_in_FrontEnd.png}
  \end{tabular}
	\caption{HCAL (phase 1 HO, sector 0) detector $\eta$ distribution in the frontend electronic coordinates.}
  \label{fig:lmapHO0EtaFEC}
 \end{center}
\end{figure}
\clearpage

\begin{figure}[htb]
 \begin{center}
  \begin{tabular}{cc}
   \includegraphics[angle=0,width=0.95\textwidth]{figures/appendix/HO0_Phi_in_FrontEnd.png}
  \end{tabular}
	\caption{HCAL (phase 1 HO, sector 0) detector $\phi$ distribution in the frontend electronic coordinates.}
  \label{fig:lmapHO0PhiFEC}
 \end{center}
\end{figure}
\clearpage

\begin{figure}[htb]
 \begin{center}
  \begin{tabular}{cc}
   \includegraphics[angle=0,width=0.95\textwidth]{figures/appendix/HO1P_Eta_in_FrontEnd.png}
  \end{tabular}
  \caption{HCAL (phase 1 HO, sector 1, plus side) detector $\eta$ distribution in the frontend electronic coordinates.}
  \label{fig:lmapHO1PEtaFEC}
 \end{center}
\end{figure}
\clearpage

\begin{figure}[htb]
 \begin{center}
  \begin{tabular}{cc}
   \includegraphics[angle=0,width=0.95\textwidth]{figures/appendix/HO1P_Phi_in_FrontEnd.png}
  \end{tabular}
  \caption{HCAL (phase 1 HO, sector 1, plus side) detector $\phi$ distribution in the frontend electronic coordinates.}
  \label{fig:lmapHO1PPhiFEC}
 \end{center}
\end{figure}
\clearpage

\begin{figure}[htb]
 \begin{center}
  \begin{tabular}{cc}
   \includegraphics[angle=0,width=0.95\textwidth]{figures/appendix/HO1M_Eta_in_FrontEnd.png}
  \end{tabular}
  \caption{HCAL (phase 1 HO, sector 1, minus side) detector $\eta$ distribution in the frontend electronic coordinates.}
  \label{fig:lmapHO1MEtaFEC}
 \end{center}
\end{figure}
\clearpage

\begin{figure}[htb]
 \begin{center}
  \begin{tabular}{cc}
   \includegraphics[angle=0,width=0.95\textwidth]{figures/appendix/HO1M_Phi_in_FrontEnd.png}
  \end{tabular}
  \caption{HCAL (phase 1 HO, sector 1, minus side) detector $\phi$ distribution in the frontend electronic coordinates.}
  \label{fig:lmapHO1MPhiFEC}
 \end{center}
\end{figure}
\clearpage

\begin{figure}[htb]
 \begin{center}
  \begin{tabular}{cc}
   \includegraphics[angle=0,width=0.95\textwidth]{figures/appendix/HO2P_Eta_in_FrontEnd.png}
  \end{tabular}
  \caption{HCAL (phase 1 HO, sector 2, plus side) detector $\eta$ distribution in the frontend electronic coordinates.}
  \label{fig:lmapHO2PEtaFEC}
 \end{center}
\end{figure}
\clearpage

\begin{figure}[htb]
 \begin{center}
  \begin{tabular}{cc}
   \includegraphics[angle=0,width=0.95\textwidth]{figures/appendix/HO2P_Phi_in_FrontEnd.png}
  \end{tabular}
  \caption{HCAL (phase 1 HO, sector 2, plus side) detector $\phi$ distribution in the frontend electronic coordinates.}
  \label{fig:lmapHO2PPhiFEC}
 \end{center}
\end{figure}
\clearpage

\begin{figure}[htb]
 \begin{center}
  \begin{tabular}{cc}
   \includegraphics[angle=0,width=0.95\textwidth]{figures/appendix/HO2M_Eta_in_FrontEnd.png}
  \end{tabular}
  \caption{HCAL (phase 1 HO, sector 2, minus side) detector $\eta$ distribution in the frontend electronic coordinates.}
  \label{fig:lmapHO2MEtaFEC}
 \end{center}
\end{figure}
\clearpage

\begin{figure}[htb]
 \begin{center}
  \begin{tabular}{cc}
   \includegraphics[angle=0,width=0.95\textwidth]{figures/appendix/HO1M_Phi_in_FrontEnd.png}
  \end{tabular}
  \caption{HCAL (phase 1 HO, sector 2, minus side) detector $\phi$ distribution in the frontend electronic coordinates.}
  \label{fig:lmapHO2MPhiFEC}
 \end{center}
\end{figure}
\clearpage

\begin{figure}[htb]
 \begin{center}
  \begin{tabular}{cc}
   \includegraphics[angle=0,width=0.95\textwidth]{figures/appendix/HO0_Crate_in_FrontEnd.png}
  \end{tabular}
  \caption{HCAL (phase 1 HO, sector 0) backend electronic coordinate crate distribution in the frontend electronic coordinates.}
  \label{fig:lmapHO0CrateFEC}
 \end{center}
\end{figure}
\clearpage

\begin{figure}[htb]
 \begin{center}
  \begin{tabular}{cc}
   \includegraphics[angle=0,width=0.95\textwidth]{figures/appendix/HO0_HTR_in_FrontEnd.png}
  \end{tabular}
  \caption{HCAL (phase 1 HO, sector 0) backend electronic coordinate HTR slot distribution in the frontend electronic coordinates.}
  \label{fig:lmapHO0HTRFEC}
 \end{center}
\end{figure}
\clearpage

\begin{figure}[htb]
 \begin{center}
  \begin{tabular}{cc}
   \includegraphics[angle=0,width=0.95\textwidth]{figures/appendix/HO0_HTR_TB_in_FrontEnd.png}
  \end{tabular}
  \caption{HCAL (phase 1 HO, sector 0) backend electronic coordinate HTR fpga distribution in the frontend electronic coordinates.}
  \label{fig:lmapHO0HTRTBFEC}
 \end{center}
\end{figure}
\clearpage

\begin{figure}[htb]
 \begin{center}
  \begin{tabular}{cc}
   \includegraphics[angle=0,width=0.95\textwidth]{figures/appendix/HO0_HTR_FI_in_FrontEnd.png}
  \end{tabular}
  \caption{HCAL (phase 1 HO, sector 0) backend electronic coordinate HTR fiber distribution in the frontend electronic coordinates.}
  \label{fig:lmapHO0HTRFIFEC}
 \end{center}
\end{figure}
\clearpage

\begin{figure}[htb]
 \begin{center}
  \begin{tabular}{cc}
   \includegraphics[angle=0,width=0.95\textwidth]{figures/appendix/HO1P_Crate_in_FrontEnd.png}
  \end{tabular}
  \caption{HCAL (phase 1 HO, sector 1, plus side) backend electronic coordinate crate distribution in the frontend electronic coordinates.}
  \label{fig:lmapHO1PCrateFEC}
 \end{center}
\end{figure}
\clearpage

\begin{figure}[htb]
 \begin{center}
  \begin{tabular}{cc}
   \includegraphics[angle=0,width=0.95\textwidth]{figures/appendix/HO1P_HTR_in_FrontEnd.png}
  \end{tabular}
  \caption{HCAL (phase 1 HO, sector 1, plus side) backend electronic coordinate HTR slot distribution in the frontend electronic coordinates.}
  \label{fig:lmapHO1PHTRFEC}
 \end{center}
\end{figure}
\clearpage

\begin{figure}[htb]
 \begin{center}
  \begin{tabular}{cc}
   \includegraphics[angle=0,width=0.95\textwidth]{figures/appendix/HO1P_HTR_TB_in_FrontEnd.png}
  \end{tabular}
  \caption{HCAL (phase 1 HO, sector 1, plus side) backend electronic coordinate HTR fpga distribution in the frontend electronic coordinates.}
  \label{fig:lmapHO1PHTRTBFEC}
 \end{center}
\end{figure}
\clearpage

\begin{figure}[htb]
 \begin{center}
  \begin{tabular}{cc}
   \includegraphics[angle=0,width=0.95\textwidth]{figures/appendix/HO1P_HTR_FI_in_FrontEnd.png}
  \end{tabular}
  \caption{HCAL (phase 1 HO, sector 1, plus side) backend electronic coordinate HTR fiber distribution in the frontend electronic coordinates.}
  \label{fig:lmapHO1PHTRFIFEC}
 \end{center}
\end{figure}
\clearpage

\begin{figure}[htb]
 \begin{center}
  \begin{tabular}{cc}
   \includegraphics[angle=0,width=0.95\textwidth]{figures/appendix/HO1M_Crate_in_FrontEnd.png}
  \end{tabular}
  \caption{HCAL (phase 1 HO, sector 1, minus side) backend electronic coordinate crate distribution in the frontend electronic coordinates.}
  \label{fig:lmapHO1MCrateFEC}
 \end{center}
\end{figure}
\clearpage

\begin{figure}[htb]
 \begin{center}
  \begin{tabular}{cc}
   \includegraphics[angle=0,width=0.95\textwidth]{figures/appendix/HO1M_HTR_in_FrontEnd.png}
  \end{tabular}
  \caption{HCAL (phase 1 HO, sector 1, minus side) backend electronic coordinate HTR slot distribution in the frontend electronic coordinates.}
  \label{fig:lmapHO1MHTRFEC}
 \end{center}
\end{figure}
\clearpage

\begin{figure}[htb]
 \begin{center}
  \begin{tabular}{cc}
   \includegraphics[angle=0,width=0.95\textwidth]{figures/appendix/HO1M_HTR_TB_in_FrontEnd.png}
  \end{tabular}
  \caption{HCAL (phase 1 HO, sector 1, minus side) backend electronic coordinate HTR fpga distribution in the frontend electronic coordinates.}
  \label{fig:lmapHO1MHTRTBFEC}
 \end{center}
\end{figure}
\clearpage

\begin{figure}[htb]
 \begin{center}
  \begin{tabular}{cc}
   \includegraphics[angle=0,width=0.95\textwidth]{figures/appendix/HO1M_HTR_FI_in_FrontEnd.png}
  \end{tabular}
  \caption{HCAL (phase 1 HO, sector 1, minus side) backend electronic coordinate HTR fiber distribution in the frontend electronic coordinates.}
  \label{fig:lmapHO1MHTRFIFEC}
 \end{center}
\end{figure}
\clearpage

\begin{figure}[htb]
 \begin{center}
  \begin{tabular}{cc}
   \includegraphics[angle=0,width=0.95\textwidth]{figures/appendix/HO2P_Crate_in_FrontEnd.png}
  \end{tabular}
  \caption{HCAL (phase 1 HO, sector 2, plus side) backend electronic coordinate crate distribution in the frontend electronic coordinates.}
  \label{fig:lmapHO2PCrateFEC}
 \end{center}
\end{figure}
\clearpage

\begin{figure}[htb]
 \begin{center}
  \begin{tabular}{cc}
   \includegraphics[angle=0,width=0.95\textwidth]{figures/appendix/HO2P_HTR_in_FrontEnd.png}
  \end{tabular}
  \caption{HCAL (phase 1 HO, sector 2, plus side) backend electronic coordinate HTR slot distribution in the frontend electronic coordinates.}
  \label{fig:lmapHO2PHTRFEC}
 \end{center}
\end{figure}
\clearpage

\begin{figure}[htb]
 \begin{center}
  \begin{tabular}{cc}
   \includegraphics[angle=0,width=0.95\textwidth]{figures/appendix/HO2P_HTR_TB_in_FrontEnd.png}
  \end{tabular}
  \caption{HCAL (phase 1 HO, sector 2, plus side) backend electronic coordinate HTR fpga distribution in the frontend electronic coordinates.}
  \label{fig:lmapHO2PHTRTBFEC}
 \end{center}
\end{figure}
\clearpage

\begin{figure}[htb]
 \begin{center}
  \begin{tabular}{cc}
   \includegraphics[angle=0,width=0.95\textwidth]{figures/appendix/HO2P_HTR_FI_in_FrontEnd.png}
  \end{tabular}
  \caption{HCAL (phase 1 HO, sector 2, plus side) backend electronic coordinate HTR fiber distribution in the frontend electronic coordinates.}
  \label{fig:lmapHO2PHTRFIFEC}
 \end{center}
\end{figure}
\clearpage

\begin{figure}[htb]
 \begin{center}
  \begin{tabular}{cc}
   \includegraphics[angle=0,width=0.95\textwidth]{figures/appendix/HO2M_Crate_in_FrontEnd.png}
  \end{tabular}
  \caption{HCAL (phase 1 HO, sector 2, minus side) backend electronic coordinate crate distribution in the frontend electronic coordinates.}
  \label{fig:lmapHO2MCrateFEC}
 \end{center}
\end{figure}
\clearpage

\begin{figure}[htb]
 \begin{center}
  \begin{tabular}{cc}
   \includegraphics[angle=0,width=0.95\textwidth]{figures/appendix/HO2M_HTR_in_FrontEnd.png}
  \end{tabular}
  \caption{HCAL (phase 1 HO, sector 2, minus side) backend electronic coordinate HTR slot distribution in the frontend electronic coordinates.}
  \label{fig:lmapHO2MHTRFEC}
 \end{center}
\end{figure}
\clearpage

\begin{figure}[htb]
 \begin{center}
  \begin{tabular}{cc}
   \includegraphics[angle=0,width=0.95\textwidth]{figures/appendix/HO2M_HTR_TB_in_FrontEnd.png}
  \end{tabular}
  \caption{HCAL (phase 1 HO, sector 2, minus side) backend electronic coordinate HTR fpga distribution in the frontend electronic coordinates.}
  \label{fig:lmapHO2MHTRTBFEC}
 \end{center}
\end{figure}
\clearpage

\begin{figure}[htb]
 \begin{center}
  \begin{tabular}{cc}
   \includegraphics[angle=0,width=0.95\textwidth]{figures/appendix/HO2M_HTR_FI_in_FrontEnd.png}
  \end{tabular}
  \caption{HCAL (phase 1 HO, sector 2, minus side) backend electronic coordinate HTR fiber distribution in the frontend electronic coordinates.}
  \label{fig:lmapHO2MHTRFIFEC}
 \end{center}
\end{figure}
\clearpage

\section{HCAL operation}


\end{document}
