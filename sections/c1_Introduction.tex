\chapter{Introduction}
Particle physics is a branch of the physics that studies the nature of the particles that constitute matter and radiation. We have a glory history of discoveries in the particle physics in the past 50 years. The standard model is a beautiful assembles of these remarkable breakthroughs. In 2012, July 4th, the discovery of the Higgs boson provides us the last piece of the puzzle.

However, the standard model is not perfect. The hierarchy issue arises with the "small" mass of the Higgs particle. The absence of the explanation in the standard model for the cold dark matter is also a blemish on the remarkable artifact.

The supersymmetric extension of the standard model is a promising solution on these issues. The hierarchy problem in the standard model can be cured with the existence of the TeV level lightest supersymmetric particle (LSP). The LSP can be neutral and stable under R-parity conservation, which provide us a best candidate on the dark matter. Therefore, the supersymmetry theory becomes one exciting theory for us to explore.

The CERN Large Hadron Collider (LHC) was designed and build with the goal of the Higgs and new physics discoveries. The LHC is designed with the highest center-of-mass energy on 14 TeV, which is the summit of the energy that human designed expriment can reach so far. This high-energy high-intensity beast provides us the opportunity to reveal the mask of new physics. The compact muon solenoid (CMS) is a general-purpose detection system installed on the collision point 5 on the LHC. The high-resolution data is taken with high efficiency in the operation. All analysis results on the CMS collaboration is based on the LHC-CMS dataset, including my analysis, targeting on the supersymmetry search in all hadronic final state.

The results of the search in missing transverse energy, extended transverse mass, b-jets and top-jets final state are demonstrated in this paper. The backgrounds are carefully estimated with robust methods in this all-hadronic final state search. In addition, our analysis group designs a customized top-jet tagger algorithm in order to obtain the relatively high efficiency in all case.

The organization of this thesis is described following. The chapter 2 gives an overview of the theory of the standard model and its supersymmetric extension, with a detailed discussion on simplified models. The LHC machine and CMS experiment are described in chapter 3. The method of the analysis and the physics interpretations are demonstrated in chapter 4. Finally, the thesis is concluded in Chapter 5.
