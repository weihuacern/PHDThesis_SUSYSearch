\chapter{Introduction}
Particle physics is a branch of the physics that studies the nature of the particles that constitute matter and radiation. There has been a glorious history of discoveries in particle physics in the past 50 years. The standard model is a beautiful assemblage of these remarkable breakthroughs. In 2012, July 4th, the discovery of the Higgs boson provided the last piece of the standard model puzzle.

However, the standard model is not perfect. The hierarchy issue arises with the "small" mass of the Higgs boson. The absence of an explanation in the standard model for cold dark matter is also a blemish on the theory.

The supersymmetric extension of the standard model is a promising solution on these issues. The hierarchy problem in the standard model can be cured with the existence of the TeV level lightest supersymmetric particle (LSP). The LSP can be neutral and stable under R-parity conservation, which provide us a candidate for dark matter. Therefore, supersymmetry represents an exciting theory for us to explore.

The CERN Large Hadron Collider (LHC) was designed and built with the goal of discovering the Higgs boson and making new-physics discoveries. The LHC has a design center-of-mass energy of 14 TeV, which is the highest energy that a human designed experiment can reach so far. This high-energy, high-intensity beast provides us the opportunity to reveal the mask of new physics. The compact muon solenoid (CMS) is a general-purpose detection system installed at collision point 5 of LHC. The high-resolution data is taken with high efficiency in the operation. All analysis results of CMS collaboration is based on the LHC-CMS data set, including my analysis, targeting on the supersymmetry search in all hadronic final state.

The results of a search in missing transverse energy, extended transverse mass, b-jets and top-quarks final state are presented in this paper. The backgrounds are carefully estimated with robust methods in this all-hadronic final state search. In addition, our analysis group designed a customized top-jet tagging algorithm in order to obtain the relatively high efficiency for all relevant values of the top quark transverse momentum.

The organization of this thesis is described as follows. Chapter 2 gives an overview of the theory of the standard model and its supersymmetric extension, with a discussion of natural SUSY and simplified models. The LHC machine and CMS experiment are described in chapter 3. The analysis method and the physics interpretations are presented in chapter 4. Finally, a conclusion is given in Chapter 5.
