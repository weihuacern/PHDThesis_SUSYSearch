\chapter{The Standard Model and The Supersymmetry Theory}
\clearpage
\section{The Standard Model}
The standard model of particle physics is a mathematical model that describes weak, electromagnetic and strong interactions (Table~\ref{tab:c24ei}) between the quarks and leptons. The standard model represents our best understanding on the subatomic world so far. The completion of standard model has begun since the middle of last century, through works of many theoretical and experimental physicists. In this section, we will have a brief review on the standard model in the first part. The challenges of the standard model will be discussed in the second half. 

\begin{table}[htbp]
\fontsize{10 pt}{1.2 em}
\selectfont
\begin{centering}
\caption{\label{tab:c24ei} The Fundamentary Interactions}
\hspace*{-4ex}
\begin{tabular}{|c|c|c|c|c|}
\hline
Interaction & Gravity & Weak & Electromagnetic & Strong \\
\hline
Source      & Mass-Energy & Flavor & Electric charge & Color charge \\
\hline
Strength    & $10^{-41}$ & $10^{-4}$ & 1 & 60 \\
\hline
Mediator    & \specialcell{Graviton\\(hypothetical)} & $W^{+},W^{-}$ and Z & Photon & Gluons \\
\hline
\end{tabular}
\par\end{centering}
\end{table}

\clearpage
\subsection{Description}

The standard model states that the material in the universe is made up of elementary fermions interacting through fields. The particles associated with the interaction fields are bosons. The standard model is described in the quantum field theory framework. The Lagrangian formalism is used in the quantum field theory to solve the motion of a particle under the influence. The expression of the standard model lagrangian is showed in the Eq~\ref{eq:c2sml}: 

\begin{equation}
%\begin{math}
%\begin{aligned}
 \begin{alignedat}{2}
  L = & -\frac{1}{4}B_{\mu\nu}B^{\mu\nu} - \frac{1}{8}tr(F_{\mu\nu}F^{\mu\nu}) - \frac{1}{2}tr(G_{\mu\nu}G^{\mu\nu}) & (Gauge \, terms) \\
      & +\begin{pmatrix} \bar{\nu}_{L} & \bar{e}_{L} \end{pmatrix}\bar{\sigma}^{\mu}iD_{\mu}\begin{pmatrix} \nu_{L} \\ e_{L} \end{pmatrix} + \bar{e}_{R}\sigma^{\mu}iD_{\mu}e_{R} + \bar{\nu}_{R}\sigma^{\mu}iD_{\mu}\nu_{R} + (h.c.) & (Lepton \, dynamical \, terms) \\
      & -\frac{\sqrt{2}}{\upsilon}[\begin{pmatrix} \bar{\nu}_{L} & \bar{e}_{L} \end{pmatrix}\phi M^{e}e_{R} + \bar{e}_{R}\bar{M}^{e}\bar{\phi}\begin{pmatrix} \nu_{L} \\ e_{L} \end{pmatrix}] & (Electron, muon, Tau \, mass \, terms) \\
      & -\frac{\sqrt{2}}{\upsilon}[\begin{pmatrix} -\bar{e}_{L} & \bar{\nu}_{L} \end{pmatrix}\phi^{*} M^{\nu}\nu_{R} + \bar{\nu}_{R}\bar{M}^{\nu}\phi^{T}\begin{pmatrix} -e_{L} \\ \nu_{L} \end{pmatrix}] & (Neutrino \, mass \, terms) \\
      & +\begin{pmatrix} \bar{u}_{L} & \bar{d}_{L} \end{pmatrix}\bar{\sigma}^{\mu}iD_{\mu}\begin{pmatrix} u_{L} \\ d_{L} \end{pmatrix} + \bar{u}_{R}\sigma^{\mu}iD_{\mu}u_{R} + \bar{d}_{R}\sigma^{\mu}iD_{\mu}d_{R} + (h.c.) & (quark \, dynamical \, terms) \\
      & -\frac{\sqrt{2}}{\upsilon}[\begin{pmatrix} \bar{u}_{L} & \bar{d}_{L} \end{pmatrix}\phi M^{d}d_{R} + \bar{d}_{R}\bar{M}^{d}\bar{\phi}\begin{pmatrix} u_{L} \\ d_{L} \end{pmatrix}] & (Down, strange, bottom \, mass \, terms) \\
      & -\frac{\sqrt{2}}{\upsilon}[\begin{pmatrix} -\bar{d}_{L} & \bar{u}_{L} \end{pmatrix}\phi^{*} M^{u}u_{R} + \bar{u}_{R}\bar{M}^{u}\phi^{T}\begin{pmatrix} -d_{L} \\ u_{L} \end{pmatrix}] & (Up, charm, top \, mass \, terms) \\
      & +\bar{D_{\mu}\phi}D^{\mu}\phi - m_{h}^{2}[\bar{\phi}\phi-\upsilon^{2}/2]^{2}/2\upsilon^{2} & (Higgs \, dynamical \, and \, mass \, terms)
 \label{eq:c2sml}
 \end{alignedat}
%\end{aligned}
%\end{math}
\end{equation}

The definition of derivative operators in the Eq~\ref{eq:c2sml} are:
\begin{equation}
 \begin{aligned}
  D_{\mu}\begin{pmatrix} \nu_{L} \\ e_{L} \end{pmatrix} = [\partial_{\mu}-\frac{ig_{1}}{2}B_{\mu}+\frac{ig_{2}}{2}W_{\mu}]\begin{pmatrix} \nu_{L} \\ e_{L} \end{pmatrix} \\
  D_{\mu}\nu_{R} = \partial_{\mu}\nu_{R},\quad D_{\mu}e_{R} = [\partial_{\mu}-ig_{1}B_{\mu}]e_{R}
 \end{aligned}
 \label{eq:c2smldl}
\end{equation}

\begin{equation}
 \begin{aligned}
  D_{\mu}\begin{pmatrix} u_{L} \\ d_{L} \end{pmatrix} = [\partial_{\mu}+\frac{ig_{1}}{6}B_{\mu}+\frac{ig_{2}}{2}W_{\mu}+igG_{\mu}]\begin{pmatrix} u_{L} \\ d_{L} \end{pmatrix} \\
  D_{\mu}u_{R} = [\partial_{\mu}+\frac{i2g_{1}}{3}B_{\mu}+igG_{\mu}]u_{R},\quad D_{\mu}d_{R} = [\partial_{\mu}-\frac{ig_{1}}{3}B_{\mu}+igG_{\mu}]d_{R}
 \end{aligned}
 \label{eq:c2smldq}
\end{equation}

\begin{equation}
 \begin{aligned}
  D_{\mu}\phi = [\partial_{\mu}+\frac{ig_{1}}{2}B_{\mu}+\frac{ig_{2}}{2}W_{\mu}]\phi
 \end{aligned}
 \label{eq:c2smldh}
\end{equation}

As we mentioned, the standard describes the strong, weak and electromagnetic interaction. These three interactions can be understood as the arising from gauge symmetries\cite{PhysRev.96.191}. Therefore, the standard model lagrangian (Eq~\ref{eq:c2sml}) is invariant under gauge transformation. However, these non-abelian gauge symmetries are conserved only with the massless gauge bosons. The gauge bosons can acquire mass through spontaneous symmetry breaking\cite{PhysRevLett.4.380,PhysRev.122.345,PhysRev.124.246}. In order to keep the gauge symmetry, a scalar boson (Higgs boson) is introduced to give mass to gauge bosons through spontaneous symmetry breaking. This is called Higgs mechanism\cite{PhysRevLett.13.321,PhysRevLett.13.508}. The last piece of the standard model, Higgs boson, has been discovered at CERN, 2012\cite{Aad:2012tfa,Chatrchyan:2012xdj}. This discovery indicates the great triumph of the standard model\cite{PhysRevLett.19.1264,PhysRevLett.30.1343,PhysRevLett.30.1346}.

According to the Eq~\ref{eq:c2sml}, if we ignore the neutrino mass term, there are 19 free parameters in the standard model. The 19 free parameters are: three lepton masses, six quark masses, three mixing angles and one CP phrase in CKM matrix\cite{PhysRevLett.10.531,Kobayashi:1973fv}, three gauge couplings, QCD vacuum angle, Higgs vacuum expectation value and Higgs mass. All the parameters have been measured directly or indirectly on different high-energy experiments. However, this is not the end of the story. 

\clearpage
\subsection{Challenges}

The standard model is great, but not perfect. There are challenges the standard model needs to face:
\begin{itemize}
\item The hierarchy problem: Another important property for the standard model we did not mention is the renormalization. A renormalizable quantum field theory permits that the perturbative calculations can only have a finite number of divergent terms in the expansion. Those divergences can be absorbed by the free parameters in the lagrangian. Therefore, the physical variable will be finite in all orders of calculation. In the standard model, the free parameters may or may not running with the evolution of renormalization group, depending on nature of the higher-order loops that need to be cancel. There are three types of particles in the standard model: fermion, vector gauge boson, and scalar Higgs boson. Higgs mass term have a quadratic divergence in the renormalization evolution. The Higgs mass (about 125GeV) input (experiment observation) has been tuned very carefully respect to Planck scale cut off ($10^{19}$ GeV) if we want to cancel the Higgs mass evolution in renormalization group. This is the infamous hierarchy problem in the standard model. The other fermions and bosons do not have this fine-tuning concern since they are protected by the symmetry. There are several approaches to solve this issue, for example, Supersymmetry (SUSY) or little higgs\cite{Schmaltz:2005ky}. 
\item The observation of cold dark matter: The existence of the cold dark matter is supported by the astrophysical observations. However, the neutrino (even with mass) in the standard model can only explain a very small fraction of cold dark matter. There are several cold dark matter candidates, such Axion for strong CP violation, neutrilino in SUSY, etc. But none of them have been detected so far.
\end{itemize}

SUSY is a promising candidate to explain both of these two problems, and will be discussed in the next section. 

\clearpage
\section{The Supersymmetric Extention of Standard Model}
The fine-tuning problem is real because the fundamental nature of the Higgs field. Therefore, a nature idea is to introduce additional symmetry to protect the Higgs field. A basic symmetry relates bosons and fermions can be introduced to solve the hierarchy problem. This is so called supersymmetry. Furthermore, in the R-parity conserved SUSY, we can have stable lightest SUSY particle with zero charge and heavy mass, which is an ideal candidate of the cold dark matter. We will use the minimal supersymmetric standard model as an example to demonstrate basics of this theory. 

\clearpage
\subsection{Minimum Supersymmetric Standard Model (MSSM)}

Minimal supersymmetric standard model (MSSM) is the simplest possible supersymmetric extension for the standard model. In MSSM, each standard model particle has a supersymmetric partner with spin vary by $\frac{1}{2}$. The particles in MSSM are listed in Table~\ref{tab:c2mssmf}.

\begin{table}[htbp]
\fontsize{10 pt}{1.2 em}
\selectfont
\begin{centering}
\caption{\label{tab:c2mssmf}List of the fields of the MSSM and their irreducible representations}
\hspace*{-4ex}
\begin{tabular}{|c|c|c|c|c|c|}
\hline
Super-multiplets & Boson field & Fermionic partners & $SU(3)_{C}$ & $SU(2)_{L}$ & $U(1)_{Y}$ \\
\hline
Gluon/Gluino     & g & $\tilde{g}$ & 8 & 1 & 0 \\
\hline
Gauge/Gaugino    & \specialcell{$W^{+},W^{-},Z$ \\ $B^{0}$} & \specialcell{$\tilde{W}^{+},\tilde{W}^{-},\tilde{Z}$ \\ $\tilde{B}^{0}$} & \specialcell{1 \\ 1} & \specialcell{3 \\ 1} & \specialcell{0 \\ 0} \\
\hline
Slepton/Lepton   & \specialcell{$(\tilde{\nu}_{e},\tilde{e})_{L}$ \\ $\tilde{e}_{R}$} & \specialcell{$(\nu_{e},e)_{L}$ \\ $e_{R}$} & \specialcell{1 \\ 1} & \specialcell{2 \\ 1} & \specialcell{-1 \\ -2} \\
\hline
Squark/Quark     & \specialcell{$(\tilde{u},\tilde{d})_{L}$ \\ $\tilde{u}_{R}$ \\ $\tilde{d}_{R}$} & \specialcell{$(u,d)_{L}$ \\ $u_{R}$ \\ $d_{R}$} & \specialcell{3 \\ 3 \\ 3} & \specialcell{2 \\ 1 \\ 1} & \specialcell{1/3 \\ 4/3 \\ -2/3} \\
\hline
Higgs/Higgsino   & \specialcell{$(H_{u}^{+},H_{u}^{0})$ \\ $(H_{d}^{0},H_{d}^{-})$} & \specialcell{$(\tilde{H}_{u}^{+},\tilde{H}_{u}^{0})$ \\ $(\tilde{H}_{d}^{0},\tilde{H}_{d}^{-})$} & \specialcell{1 \\ 1} & \specialcell{2 \\ 2} & \specialcell{1 \\ -1} \\
\hline
\end{tabular}
\par\end{centering}
\end{table}

There are two important features in MSSM, the first one is spontaneously supersymmetry broken. The particles in the same super partner pair must have same mass if unbroken supersymmetry is unbroken. However, no super partners have been observed so far. The spontaneously supersymmetry broken is introduced to explain this difference. The heavy sparticles emerge from the supersymmetry breaking. Therefore, the Higgs mass correction can be expressed by Eq~\ref{eq:c2mssmhmc}:

\begin{equation}
 \Delta m_{H}^{2} = \frac{1}{8\pi}(\lambda_{S}-|\lambda_{f}|^{2})\Lambda_{UV}+m_{s}^{2}(\frac{\lambda}{16\pi^{2}}ln(\Lambda_{UV}/m_{s}))+...
 \label{eq:c2mssmhmc}
\end{equation}

The relation $\lambda_{S}=|\lambda_{f}|^{2}$ occurs in the unbroken supersymmetry and still be held in the soft supersymmetry breaking\cite{Martin:1997ns}. Therefore, the quadratic sensitivity on Higgs mass for high mass scale disappears.

The other property is the R-parity. The R-parity is defined as $P_{R}=(-1)^{3(B-L)+3s}$, where B for baryon number, L for lepton number and s for spin. Therefore, all the standard model particles have R-parity +1, while supersymmetric particles have R-parity -1. The lightest supersymmetry particle is stable if the R-parity is conserved. As a result, the gauginos, higgsinos and sneutrinos can be the cold dark matter candidates. This feature provides one motivation for searching the gauginos (also called neutralino) in the R-parity conserved SUSY model. 

\clearpage
\subsection{SUSY naturalness}

The naturalness of a theory means that the ratios between free parameters or physical constants in this theory should be in order 1, so the parameters are not fine-tuned. In the hierarchy problem, the scalar Higgs mass is fine tuned without symmetry protection. We can define a measure of the fine-tuning by Eq~\ref{eq:c2nsusymeasure}:
\begin{equation}
 \Delta = \frac{2\delta m_{H}^{2}}{m_{h}^{2}}
 \label{eq:c2nsusymeasure}
\end{equation}
The $m_{h}$ is the physical neutral CP-even Higgs mass, and the $m_{H}$ is a general linear combination of the various masses of the Higgs fields with coefficient depend on the mixing angle ($\beta$ in MSSM). This fine-tuning measure can be used for the sparticle mass constrain. 

Let’s take stop mass constrain as an example. The Higgs potential in SUSY is corrected by both gauge and Yukawa interactions. The major contribution comes from top-stop loop. The Higgs mass correction from stop loop can be expressed by Eq~\ref{eq:c2nsusystoptree}: 
\begin{equation}
 \delta m_{H_{u}}^{2}|_{stop} = - \frac{3}{8\pi^{2}}y_{t}^{2}(m_{Q_{3}}^{2}+m_{u_{3}}^{2}+|A_{t}|^{2})log(\frac{\Lambda_{UV}}{TeV})
 \label{eq:c2nsusystoptree}
\end{equation}

We can eewrite it by the stop mass eigenvalues in Eq~\ref{eq:c2nsusystopeigen}:
\begin{equation}
	\delta m_{H_{u}}^{2}|_{stop} \approx - \frac{3}{8\pi^{2}}y_{t}^{2}(m_{\tilde{t_{1}}}^{2}+m_{\tilde{t_{2}}}^{2}-2m_{t}^{2}+\frac{m_{\tilde{t_{1}}}^{2}-m_{\tilde{t_{2}}}^{2}}{m_{t}^{2}}cos^{2}\theta_{\tilde{t}}sin^{2}\theta_{\tilde{t}})log(\frac{\Lambda_{UV}}{TeV})
 \label{eq:c2nsusystopeigen}
\end{equation}

The requirement of a natural Higgs potential sets an upper bound on the stop mass\cite{Papucci:2011wy}, described by Eq~\ref{eq:c2nsusystopbound}:

\begin{equation}
 \sqrt{m_{\tilde{t_{1}}}^{2}+m_{\tilde{t_{2}}}^{2}} \leq 600GeV\frac{sin\beta}{(1+x_{t})^{1/2}} (\frac{log(\Lambda/TeV)}{3})^{-1/2}(\frac{m_{h}}{120GeV})(\frac{\Delta^{-1}}{20\%})^{-1/2}
 \label{eq:c2nsusystopbound}
\end{equation}

We can set the upper bound for gluino and higgsino mass in the similar way\cite{Papucci:2011wy}. To summarize, the requirements for the natural SUSY are:
\begin{itemize}
\item Tow stops and one left-handed sbottom, both below 500-700 GeV
\item Two higgsinos, below 200-350 GeV
\item Not too heavy gluino, below 900-1500GeV
\end{itemize}

Therefore, the stop, sbottom, higgsino and gluino are the interested search target in natural SUSY. Higgsinos are produced through electroweak process. The cross section is relatively low on the large hadron collider. The light stops have the leading contribution on the Higgs mass correction. Therefore, the search for stop and gluino is a desirable target on the large hadron collider (LHC). We have the most probability to find stop and gluino in hadronic channel since the hadronic process cross section is relatively high on the LHC. 

Then we need to consider the R-parity. The R-parity violated (PRV) SUSY model can also be a natural SUSY model. However, since the SUSY LSP can decay into standard model particle, there is no cold dark matter candidate in RPV SUSY. 

Now, we can put all fact together to obtain a motivated search strategy: 
\begin{itemize}
\item Naturalness requests: stop, sbottom, gluino and higgsino
\item Cold dark matter candidate: R-parity conservation
\item Experiment sensitivity: Large cross section in hadronic production on hadron collider
\end{itemize}

Therefore, the idea “the stop and gluino search in hadronic channel in hadronic channel” is formed after put all thoughts together.

\clearpage
\subsection{SUSY simplified model}

