\chapter{The standard model and supersymmetry}
\clearpage
\section{The standard model}
The standard model of particle physics is a mathematical model that describes weak, electromagnetic and strong interactions (Table~\ref{tab:c24ei}) between the quarks and leptons. The standard model represents our best understanding of the subatomic world so far. The development of the standard model has been ongoing since the middle of the last century, through the work of many theoretical and experimental physicists. In the first part of this chapter, we will have a brief review of the standard model. In the second part, the challenges of the standard model will be discussed. 

\begin{table}[htbp]
\fontsize{10 pt}{1.2 em}
\selectfont
\begin{centering}
\caption{\label{tab:c24ei} The Fundamentary Interactions}
\hspace*{-4ex}
\begin{tabular}{|c|c|c|c|c|}
\hline
Interaction                                      & Gravity & Weak & Electromagnetic & Strong \\
\hline
Source                                           & Mass-Energy & Flavor & Electric charge & Color charge \\
\hline
\specialcell{Strength at the \\ scale of quarks} & $10^{-41}$ & $10^{-4}$ & 1 & 60 \\
\hline
Mediator                                         & \specialcell{Graviton\\(hypothetical)} & $W^{+},W^{-}$ and Z & Photon & Gluons \\
\hline
\end{tabular}
\par\end{centering}
\end{table}

\clearpage
\subsection{Description}

The standard model states that the material in the universe is made up of elementary fermions interacting through fields. The particles associated with the interaction fields are bosons. The standard model is described in quantum field theory framework. The Lagrangian formalism is used in the quantum field theory to describe the motion of a particle under the influence of field. The standard model Lagrangian is shown in Eq~\ref{eq:c2sml}: 

\begin{equation}
%\begin{math}
%\begin{aligned}
 \begin{alignedat}{2}
  L = & -\frac{1}{4}B_{\mu\nu}B^{\mu\nu} - \frac{1}{8}tr(F_{\mu\nu}F^{\mu\nu}) - \frac{1}{2}tr(G_{\mu\nu}G^{\mu\nu}) & (Gauge \, terms) \\
      & +\begin{pmatrix} \bar{\nu}_{L} & \bar{e}_{L} \end{pmatrix}\bar{\sigma}^{\mu}iD_{\mu}\begin{pmatrix} \nu_{L} \\ e_{L} \end{pmatrix} + \bar{e}_{R}\sigma^{\mu}iD_{\mu}e_{R} + \bar{\nu}_{R}\sigma^{\mu}iD_{\mu}\nu_{R} + (h.c.) & (Lepton \, dynamical \, terms) \\
      & -\frac{\sqrt{2}}{\upsilon}[\begin{pmatrix} \bar{\nu}_{L} & \bar{e}_{L} \end{pmatrix}\phi M^{e}e_{R} + \bar{e}_{R}\bar{M}^{e}\bar{\phi}\begin{pmatrix} \nu_{L} \\ e_{L} \end{pmatrix}] & (Electron, muon, Tau \, mass \, terms) \\
      & -\frac{\sqrt{2}}{\upsilon}[\begin{pmatrix} -\bar{e}_{L} & \bar{\nu}_{L} \end{pmatrix}\phi^{*} M^{\nu}\nu_{R} + \bar{\nu}_{R}\bar{M}^{\nu}\phi^{T}\begin{pmatrix} -e_{L} \\ \nu_{L} \end{pmatrix}] & (Neutrino \, mass \, terms) \\
      & +\begin{pmatrix} \bar{u}_{L} & \bar{d}_{L} \end{pmatrix}\bar{\sigma}^{\mu}iD_{\mu}\begin{pmatrix} u_{L} \\ d_{L} \end{pmatrix} + \bar{u}_{R}\sigma^{\mu}iD_{\mu}u_{R} + \bar{d}_{R}\sigma^{\mu}iD_{\mu}d_{R} + (h.c.) & (quark \, dynamical \, terms) \\
      & -\frac{\sqrt{2}}{\upsilon}[\begin{pmatrix} \bar{u}_{L} & \bar{d}_{L} \end{pmatrix}\phi M^{d}d_{R} + \bar{d}_{R}\bar{M}^{d}\bar{\phi}\begin{pmatrix} u_{L} \\ d_{L} \end{pmatrix}] & (Down, strange, bottom \, mass \, terms) \\
      & -\frac{\sqrt{2}}{\upsilon}[\begin{pmatrix} -\bar{d}_{L} & \bar{u}_{L} \end{pmatrix}\phi^{*} M^{u}u_{R} + \bar{u}_{R}\bar{M}^{u}\phi^{T}\begin{pmatrix} -d_{L} \\ u_{L} \end{pmatrix}] & (Up, charm, top \, mass \, terms) \\
      & +\bar{D_{\mu}\phi}D^{\mu}\phi - m_{h}^{2}[\bar{\phi}\phi-\upsilon^{2}/2]^{2}/2\upsilon^{2} & (Higgs \, dynamical \, and \, mass \, terms)
 \label{eq:c2sml}
 \end{alignedat}
%\end{aligned}
%\end{math}
\end{equation}

The definition of derivative operators in the Eq~\ref{eq:c2sml} are:
\begin{equation}
 \begin{aligned}
  D_{\mu}\begin{pmatrix} \nu_{L} \\ e_{L} \end{pmatrix} = [\partial_{\mu}-\frac{ig_{1}}{2}B_{\mu}+\frac{ig_{2}}{2}W_{\mu}]\begin{pmatrix} \nu_{L} \\ e_{L} \end{pmatrix} \\
  D_{\mu}\nu_{R} = \partial_{\mu}\nu_{R},\quad D_{\mu}e_{R} = [\partial_{\mu}-ig_{1}B_{\mu}]e_{R}
 \end{aligned}
 \label{eq:c2smldl}
\end{equation}

\begin{equation}
 \begin{aligned}
  D_{\mu}\begin{pmatrix} u_{L} \\ d_{L} \end{pmatrix} = [\partial_{\mu}+\frac{ig_{1}}{6}B_{\mu}+\frac{ig_{2}}{2}W_{\mu}+igG_{\mu}]\begin{pmatrix} u_{L} \\ d_{L} \end{pmatrix} \\
  D_{\mu}u_{R} = [\partial_{\mu}+\frac{i2g_{1}}{3}B_{\mu}+igG_{\mu}]u_{R},\quad D_{\mu}d_{R} = [\partial_{\mu}-\frac{ig_{1}}{3}B_{\mu}+igG_{\mu}]d_{R}
 \end{aligned}
 \label{eq:c2smldq}
\end{equation}

\begin{equation}
 \begin{aligned}
  D_{\mu}\phi = [\partial_{\mu}+\frac{ig_{1}}{2}B_{\mu}+\frac{ig_{2}}{2}W_{\mu}]\phi
 \end{aligned}
 \label{eq:c2smldh}
\end{equation}

As mentioned, the standard model describes the strong, weak and electromagnetic interaction. These three interactions can be understood as the arising from gauge symmetries\cite{PhysRev.96.191}. Therefore, the standard model Lagrangian (Eq~\ref{eq:c2sml}) is invariant under gauge transformation. However, these non-Abelian gauge symmetries are conserved only with the massless gauge bosons. The gauge bosons can acquire mass through spontaneous symmetry breaking\cite{PhysRevLett.4.380,PhysRev.122.345,PhysRev.124.246}. In order to preserve the gauge symmetry, a scalar boson (Higgs boson) is introduced to give mass to gauge bosons through spontaneous symmetry breaking. This is called the Higgs mechanism\cite{PhysRevLett.13.321,PhysRevLett.13.508}. The last piece of the standard model, the Higgs boson, has been discovered at CERN in 2012\cite{Aad:2012tfa,Chatrchyan:2012xdj}. This discovery represents a the great triumph of the standard model\cite{PhysRevLett.19.1264,PhysRevLett.30.1343,PhysRevLett.30.1346}.

According to the Eq~\ref{eq:c2sml}, if we ignore the neutrino mass term, there are 19 free parameters in the standard model. The 19 free parameters are: three lepton masses, six quark masses, three mixing angles and one CP phrase in CKM matrix\cite{PhysRevLett.10.531,Kobayashi:1973fv}, three gauge couplings, QCD vacuum angle, Higgs vacuum expectation value and Higgs boson mass. All the parameters have been measured directly or indirectly on different high-energy experiments. However, this is not the end of the story. 

\clearpage
\subsection{Challenges}

The standard model is very successful, but not perfect. There are challenges the standard model does not resolve:
\begin{itemize}
\item The hierarchy problem: Another important property of the standard model is renormalizability. A renormalizable quantum field theory has only a finite number of divergent terms in the expansion of a perturbative calculation. Those divergences can be absorbed by the free parameters in the Lagrangian. Therefore, the physical variable will be finite in all orders of calculation. In the standard model, the free parameters may or may not run with the evolution of the renormalization group, depending on nature of the higher-order loops. There are three types of particles in the standard model: fermions, vector gauge bosons, and the scalar Higgs boson. Higgs boson mass term have a quadratic divergence in the renormalization evolution. The Higgs boson mass (about 125 GeV) input (experimental observation) has been to be tuned very carefully with respect to the Planck scale cut off ($10^{19}$ GeV) if we want to cancel the Higgs boson mass evolution in renormalization group. This is the infamous hierarchy problem in the standard model. The other fermions and bosons do not have this fine-tuning concern since they are protected by the symmetries. There are several approaches to solve this issue, for example, Supersymmetry (SUSY) or little Higgs models\cite{Schmaltz:2005ky}. 
\item The observation of cold dark matter: The existence of cold dark matter is supported by astrophysical observations. However, the neutrino (even with mass) in the standard model can only explain a very small fraction of dark matter. There are several cold dark matter candidates, such the Axion introduced to explain strong CP violation, the neutrilino in SUSY, etc. But none of them have been detected so far.
\end{itemize}

SUSY is a promising candidate to explain both of these two problems, and will be discussed in the next section. 

\clearpage
\section{The supersymmetric extentsion of the standard model}
A natural idea to solve the fine-tuning problem is to introduce a symmetry to protect the Higgs field. A basic symmetry relating bosons and fermions can be introduced to solve the hierarchy problem. This is so-called supersymmetry. Furthermore, in R-parity\cite{Martin:1997ns} conserved SUSY, we can have a stable, neutral, massive lightest SUSY particle, which is an ideal candidate for cold dark matter. We will use the minimal supersymmetric standard model as an example to demonstrate the basics of this theory. 

\clearpage
\subsection{Minimum supersymmetric standard model (MSSM)}

The minimal supersymmetric standard model (MSSM) is the simplest possible supersymmetric extension of the standard model. In the MSSM, each standard model particle has a supersymmetric partner with the same quantum numbers except for spin, which differs by $\frac{1}{2}$. The particles in the MSSM are listed in Table~\ref{tab:c2mssmf}.

\begin{table}[htbp]
\fontsize{10 pt}{1.2 em}
\selectfont
\begin{centering}
\caption{\label{tab:c2mssmf}List of the fields of the MSSM and their irreducible representations}
\hspace*{-4ex}
\begin{tabular}{|c|c|c|c|c|c|}
\hline
Super-multiplets & Boson field & Fermionic partners & $SU(3)_{C}$ & $SU(2)_{L}$ & $U(1)_{Y}$ \\
\hline
Gluon/Gluino     & g & $\tilde{g}$ & 8 & 1 & 0 \\
\hline
Gauge/Gaugino    & \specialcell{$W^{+},W^{-},Z$ \\ $B^{0}$} & \specialcell{$\tilde{W}^{+},\tilde{W}^{-},\tilde{Z}$ \\ $\tilde{B}^{0}$} & \specialcell{1 \\ 1} & \specialcell{3 \\ 1} & \specialcell{0 \\ 0} \\
\hline
Slepton/Lepton   & \specialcell{$(\tilde{\nu}_{e},\tilde{e})_{L}$ \\ $\tilde{e}_{R}$} & \specialcell{$(\nu_{e},e)_{L}$ \\ $e_{R}$} & \specialcell{1 \\ 1} & \specialcell{2 \\ 1} & \specialcell{-1 \\ -2} \\
\hline
Squark/Quark     & \specialcell{$(\tilde{u},\tilde{d})_{L}$ \\ $\tilde{u}_{R}$ \\ $\tilde{d}_{R}$} & \specialcell{$(u,d)_{L}$ \\ $u_{R}$ \\ $d_{R}$} & \specialcell{3 \\ 3 \\ 3} & \specialcell{2 \\ 1 \\ 1} & \specialcell{1/3 \\ 4/3 \\ -2/3} \\
\hline
Higgs/Higgsino   & \specialcell{$(H_{u}^{+},H_{u}^{0})$ \\ $(H_{d}^{0},H_{d}^{-})$} & \specialcell{$(\tilde{H}_{u}^{+},\tilde{H}_{u}^{0})$ \\ $(\tilde{H}_{d}^{0},\tilde{H}_{d}^{-})$} & \specialcell{1 \\ 1} & \specialcell{2 \\ 2} & \specialcell{1 \\ -1} \\
\hline
\end{tabular}
\par\end{centering}
\end{table}

There are two important features of the MSSM. The first is the spontaneious breaking of supersymmetry. The particles in the same super-partner pair must have the same mass in unbroken supersymmetry. However, no super partners have been observed so far. Spontaneously broken supersymmetry is introduced to explain this difference. Heavy sparticles, the super-partners of the standard model particles, emerge from the supersymmetry breaking. Therefore, the Higgs boson mass correction can be expressed by Eq~\ref{eq:c2mssmhmc}:

\begin{equation}
 \Delta m_{H}^{2} = \frac{1}{8\pi}(\lambda_{S}-|\lambda_{f}|^{2})\Lambda_{UV}+m_{s}^{2}(\frac{\lambda}{16\pi^{2}}ln(\Lambda_{UV}/m_{s}))+...
 \label{eq:c2mssmhmc}
\end{equation}

The relation $\lambda_{S}=|\lambda_{f}|^{2}$ occurs in unbroken supersymmetry and still holds in soft supersymmetry breaking\cite{Martin:1997ns}. Therefore, the quadratic sensitivity of the Higgs boson mass disappears.

The other property is R-parity. The R-parity is defined as $P_{R}=(-1)^{3(B-L)+3s}$, where B is the baryon number, L is the lepton number and s is the spin. All the standard model particles have R-parity +1, while supersymmetric particles have R-parity -1. The lightest supersymmetry particle is stable if R-parity is conserved. As a result, the gauginos, higgsinos and sneutrinos can be the cold dark matter candidates. This feature provides a strong motivation to search for the gauginos (also called neutralino) in the R-parity conserved SUSY model. 

\clearpage
\subsection{SUSY naturalness}

The naturalness of a theory means that the ratios between free parameters or physical constants in this theory should be in order 1, so the parameters are not fine-tuned. In the hierarchy problem, the scalar Higgs boson mass is fine tuned without symmetry protection. We can define a measure of the fine-tuning by Eq~\ref{eq:c2nsusymeasure}:
\begin{equation}
 \Delta = \frac{2\delta m_{H}^{2}}{m_{h}^{2}}
 \label{eq:c2nsusymeasure}
\end{equation}
The $m_{h}$ is the physical neutral CP-even Higgs boson mass, and the $m_{H}$ is a general linear combination of the various masses of the Higgs fields with coefficient depend on the mixing angle ($\beta$ in MSSM). This fine-tuning measure can be used for the sparticle mass constrain. 

One can take the top squark mass constraint as an example. The Higgs boson potential in SUSY is corrected by both gauge and Yukawa interactions. The major contribution comes from the top-top squark loop. The Higgs boson mass correction from the top squark loop can be expressed by Eq~\ref{eq:c2nsusystoptree}: 
\begin{equation}
 \delta m_{H_{u}}^{2}|_{top squark} = - \frac{3}{8\pi^{2}}y_{t}^{2}(m_{Q_{3}}^{2}+m_{u_{3}}^{2}+|A_{t}|^{2})log(\frac{\Lambda_{UV}}{TeV})
 \label{eq:c2nsusystoptree}
\end{equation}

We can reexpress this result using top squark mass eigenvalues in Eq~\ref{eq:c2nsusystopeigen}:
\begin{equation}
	\delta m_{H_{u}}^{2}|_{top squark} \approx - \frac{3}{8\pi^{2}}y_{t}^{2}(m_{\tilde{t_{1}}}^{2}+m_{\tilde{t_{2}}}^{2}-2m_{t}^{2}+\frac{m_{\tilde{t_{1}}}^{2}-m_{\tilde{t_{2}}}^{2}}{m_{t}^{2}}cos^{2}\theta_{\tilde{t}}sin^{2}\theta_{\tilde{t}})log(\frac{\Lambda_{UV}}{TeV})
 \label{eq:c2nsusystopeigen}
\end{equation}

The requirement of a natural Higgs boson potential sets an upper bound on the top squark mass\cite{Papucci:2011wy}, described by Eq~\ref{eq:c2nsusystopbound}:

\begin{equation}
 \sqrt{m_{\tilde{t_{1}}}^{2}+m_{\tilde{t_{2}}}^{2}} \leq 600~GeV\frac{sin\beta}{(1+x_{t})^{1/2}} (\frac{log(\Lambda/TeV)}{3})^{-1/2}(\frac{m_{h}}{120~GeV})(\frac{\Delta^{-1}}{20\%})^{-1/2}
 \label{eq:c2nsusystopbound}
\end{equation}

We can set the upper bound for gluino and higgsino mass in similar way\cite{Papucci:2011wy}. The higgsino and gluino mass upper bounds can be described by Eq~\ref{eq:c2nsusystopbound} and Eq~\ref{eq:c2nsusygluinobound} respectively.

\begin{equation}
 \mu \leq 200~GeV (\frac{m_{h}}{120~GeV})(\frac{\Delta^{-1}}{20\%})^{-1/2}
 \label{eq:c2nsusyhiggsinobound}
\end{equation}

\begin{equation}
 M_{3} \leq 900~GeV sin\beta (\frac{log(\Lambda/TeV)}{3})^{-1}(\frac{m_{h}}{120~GeV})(\frac{\Delta^{-1}}{20\%})^{-1/2}
 \label{eq:c2nsusygluinobound}
\end{equation}

To summarize, the requirements for the natural SUSY are:
\begin{itemize}
\item Tow top squarks and one left-handed bottom squark, both below 500-700 GeV. 
The "left-handed" sbottom is in the same SU(2)
weak multiplet as the "left-handed" stop and
thereby get a mass constraint from naturalness,
since the masses of particles in the same multiplet
should not be too different. Since the "right-handed"
sbottom is not in an SU(2) weak multiplet with a
stop, there is no naturalness mass constraint on
the right-handed sbottom.
\item Two higgsinos, below 200-350 GeV.
\item Not too heavy gluino, below 900-1500 GeV.
\end{itemize}

Therefore, the top squarks, bottom squark, higgsino and gluino are the interesting search target in natural SUSY. The top squarks provide the leading contribution to the Higgs boson mass correction. We have the most probability to find top squark and gluino in hadronic channel since the hadronic process cross section is relatively high at the LHC. 

Then we need to consider R-parity. R-parity violating (PRV) SUSY models can also be natural SUSY models. However, since the SUSY LSP can decay into standard model particle in RPV models. There is no cold dark matter candidate in RPV SUSY. 

Now, we can put all facts together to motivate a search strategy: 
\begin{itemize}
\item Naturalness suggests: top squarks, bottom squark, gluino and higgsino
\item Cold dark matter candidate: R-parity conservation
\item Experimental sensitivity: Large cross section in hadronic production on hadron collider gluino and top squark production
\end{itemize}

Therefore, the idea “the top squark and gluino search in hadronic channel in hadronic channel” is formed after put all thoughts together. This motivates a search for gluinos and top squarks.  The search is performed in all-hadronic final states since this final state has the largest expected branching fraction.

\clearpage
\subsection{SUSY simplified model}
As mentioned in the previous section, hadronic decayed top squarks or gluino are ideal SUSY search channels at the LHC. However, there are too many free parameters in MSSM. Setting limits on the parameters can be tricky when there is a high degree of freedom in the parameter phrase space. 

Simplified models\cite{Alwall:2008ag} provides a solution to this issue. A simplified model represents a specific event topology, with SUSY mass values treated as free parameters. They usually contain only 2-4 free parameters (the masses of the sparticles), making it much simpler to set limit. Simplified models are widely used in SUSY searches at the LHC\cite{CMS-SMS-paper}.

In this thesis, the model topologies we are interested in are top squarks pair production with decays of the top squarks into top and LSP (T2tt), gluino pair production with three-body decay of the gluinos into a top quark-antiquark pair and LSP (T1 models), and gluino pair production followed by the decay of the gluinos into a top quark and a top squark, followed by the decay of the top squark to a top quark and the LSP (T5 models). The signal topologies will be discussed again in the chapter 4.

Now, we have a motivated SUSY strategy and an effective interpretation method. Before discussing the analysis, we need to have a detour to explain the experimental instruments and the basic physics object definitions. This is done in the next chapter. 
